\documentclass[10pt,ignorenonframetext,compress, aspectratio=169]{beamer}
\setbeamertemplate{caption}[numbered]
\setbeamertemplate{caption label separator}{: }
\setbeamercolor{caption name}{fg=normal text.fg}
\beamertemplatenavigationsymbolsempty
\usepackage{lmodern}
\usepackage{amssymb,amsmath,mathtools}
\usepackage{ifxetex,ifluatex}
\usepackage{fixltx2e} % provides \textsubscript
\ifnum 0\ifxetex 1\fi\ifluatex 1\fi=0 % if pdftex
  \usepackage[T1]{fontenc}
  \usepackage[utf8]{inputenc}
\else % if luatex or xelatex
  \ifxetex
    \usepackage{mathspec}
  \else
    \usepackage{fontspec}
  \fi
  %%\defaultfontfeatures{Ligatures=TeX,Scale=MatchLowercase}
  \defaultfontfeatures{Scale=MatchLowercase}
\fi
\usetheme[]{metropolis}
% use upquote if available, for straight quotes in verbatim environments
\IfFileExists{upquote.sty}{\usepackage{upquote}}{}
% use microtype if available
\IfFileExists{microtype.sty}{%
\usepackage{microtype}
\UseMicrotypeSet[protrusion]{basicmath} % disable protrusion for tt fonts
}{}
\newif\ifbibliography
\usepackage{color}
\usepackage{fancyvrb}
\newcommand{\VerbBar}{|}
\newcommand{\VERB}{\Verb[commandchars=\\\{\}]}
\DefineVerbatimEnvironment{Highlighting}{Verbatim}{commandchars=\\\{\}}
% Add ',fontsize=\small' for more characters per line
\usepackage{framed}
\definecolor{shadecolor}{RGB}{248,248,248}
\newenvironment{Shaded}{\begin{snugshade}}{\end{snugshade}}
\newcommand{\KeywordTok}[1]{\textcolor[rgb]{0.13,0.29,0.53}{\textbf{{#1}}}}
\newcommand{\DataTypeTok}[1]{\textcolor[rgb]{0.13,0.29,0.53}{{#1}}}
\newcommand{\DecValTok}[1]{\textcolor[rgb]{0.00,0.00,0.81}{{#1}}}
\newcommand{\BaseNTok}[1]{\textcolor[rgb]{0.00,0.00,0.81}{{#1}}}
\newcommand{\FloatTok}[1]{\textcolor[rgb]{0.00,0.00,0.81}{{#1}}}
\newcommand{\ConstantTok}[1]{\textcolor[rgb]{0.00,0.00,0.00}{{#1}}}
\newcommand{\CharTok}[1]{\textcolor[rgb]{0.31,0.60,0.02}{{#1}}}
\newcommand{\SpecialCharTok}[1]{\textcolor[rgb]{0.00,0.00,0.00}{{#1}}}
\newcommand{\StringTok}[1]{\textcolor[rgb]{0.31,0.60,0.02}{{#1}}}
\newcommand{\VerbatimStringTok}[1]{\textcolor[rgb]{0.31,0.60,0.02}{{#1}}}
\newcommand{\SpecialStringTok}[1]{\textcolor[rgb]{0.31,0.60,0.02}{{#1}}}
\newcommand{\ImportTok}[1]{{#1}}
\newcommand{\CommentTok}[1]{\textcolor[rgb]{0.56,0.35,0.01}{\textit{{#1}}}}
\newcommand{\DocumentationTok}[1]{\textcolor[rgb]{0.56,0.35,0.01}{\textbf{\textit{{#1}}}}}
\newcommand{\AnnotationTok}[1]{\textcolor[rgb]{0.56,0.35,0.01}{\textbf{\textit{{#1}}}}}
\newcommand{\CommentVarTok}[1]{\textcolor[rgb]{0.56,0.35,0.01}{\textbf{\textit{{#1}}}}}
\newcommand{\OtherTok}[1]{\textcolor[rgb]{0.56,0.35,0.01}{{#1}}}
\newcommand{\FunctionTok}[1]{\textcolor[rgb]{0.00,0.00,0.00}{{#1}}}
\newcommand{\VariableTok}[1]{\textcolor[rgb]{0.00,0.00,0.00}{{#1}}}
\newcommand{\ControlFlowTok}[1]{\textcolor[rgb]{0.13,0.29,0.53}{\textbf{{#1}}}}
\newcommand{\OperatorTok}[1]{\textcolor[rgb]{0.81,0.36,0.00}{\textbf{{#1}}}}
\newcommand{\BuiltInTok}[1]{{#1}}
\newcommand{\ExtensionTok}[1]{{#1}}
\newcommand{\PreprocessorTok}[1]{\textcolor[rgb]{0.56,0.35,0.01}{\textit{{#1}}}}
\newcommand{\AttributeTok}[1]{\textcolor[rgb]{0.77,0.63,0.00}{{#1}}}
\newcommand{\RegionMarkerTok}[1]{{#1}}
\newcommand{\InformationTok}[1]{\textcolor[rgb]{0.56,0.35,0.01}{\textbf{\textit{{#1}}}}}
\newcommand{\WarningTok}[1]{\textcolor[rgb]{0.56,0.35,0.01}{\textbf{\textit{{#1}}}}}
\newcommand{\AlertTok}[1]{\textcolor[rgb]{0.94,0.16,0.16}{{#1}}}
\newcommand{\ErrorTok}[1]{\textcolor[rgb]{0.64,0.00,0.00}{\textbf{{#1}}}}
\newcommand{\NormalTok}[1]{{#1}}
\usepackage{longtable,booktabs}
\usepackage{caption}
% These lines are needed to make table captions work with longtable:
\makeatletter
\def\fnum@table{\tablename~\thetable}
\makeatother

% Prevent slide breaks in the middle of a paragraph:
\widowpenalties 1 10000
\raggedbottom

\AtBeginPart{
  \let\insertpartnumber\relax
  \let\partname\relax
  \frame{\partpage}
}
\AtBeginSection{
  \ifbibliography
  \else
    \let\insertsectionnumber\relax
    \let\sectionname\relax
    \frame{\sectionpage}
  \fi
}
\AtBeginSubsection{
  \let\insertsubsectionnumber\relax
  \let\subsectionname\relax
  \frame{\subsectionpage}
}

\setlength{\parindent}{0pt}
\setlength{\parskip}{6pt plus 2pt minus 1pt}
\setlength{\emergencystretch}{3em}  % prevent overfull lines
\providecommand{\tightlist}{%
  \setlength{\itemsep}{0pt}\setlength{\parskip}{0pt}}
\setcounter{secnumdepth}{0}

%% GLS Added
% Textcomp for various common symbols
\usepackage{textcomp}

\usepackage{booktabs}

% Creative Commons Icons
\usepackage[scale=1]{ccicons}

\newenvironment{centrefig}{\begin{figure}\centering}{\end{figure}}
\newcommand{\columnsbegin}{\begin{columns}}
\newcommand{\columnsend}{\end{columns}}
\newcommand{\centreFigBegin}{\begin{figure}\centering}
\newcommand{\centreFigEnd}{\end{figure}}
%%

\DefineVerbatimEnvironment{Highlighting}{Verbatim}{commandchars=\\\{\}, fontsize=\tiny}
% make console-output smaller:
\makeatletter
\def\verbatim{\tiny\@verbatim \frenchspacing\@vobeyspaces \@xverbatim}
\makeatother
\setlength{\parskip}{0pt}
\setlength{\OuterFrameSep}{-4pt} % was -4pt
\makeatletter
\preto{\@verbatim}{\topsep=-10pt \partopsep=-10pt} % were -10pt
\makeatother

\title{Learning R}
\author{Gavin L. Simpson}
\date{February, 2017}

\begin{document}
\frame{\titlepage}

\section{Introduction to R}\label{introduction-to-r}

\begin{frame}{Why R?}

\begin{itemize}
\tightlist
\item
  R was designed from the ground up as a language for data analysis
\item
  It is free and open source, and available on all the major OSes
\item
  Huge package ecosystem (\textgreater{} 10,000) covering bewildering
  array of statistical methods, data visualisations, data import and
  manipulation
\item
  Cutting edge; R is used by thousands of statisticians \& R code or a
  package often accompanies papers developing new methods
\item
  For the most part, a great community providing help, blog posts etc
\end{itemize}

\end{frame}

\begin{frame}{What is R?}

\begin{itemize}
\tightlist
\item
  The S statistical language was started at Bell Labs on May 5, 1976
\item
  A system for general data analysis jobs that could replace the
  \emph{ad hoc} creation of Fortran applications
\item
  The S language was licensed by Insightful Corporation for use in their
  \emph{S-PLUS} software
\item
  In 2004 Insightful bought the S language from Lucent (formerly AT\&T
  and before that Bell Labs)
\item
  Robert Gentleman and Ross Ihaka designed a language that was
  compatible with S but which worked in a different way internally
\item
  They called this language R
\item
  There was a lot of interest in R and eventually it was made Open
  Source under the GNU GPL-2
\item
  R has drawn around it a group of dedicated stewards of the R software
  --- \alert{R Core}
\end{itemize}

\end{frame}

\begin{frame}{R on the web}

\begin{itemize}
\tightlist
\item
  The R homepage is located at:
  \href{http://www.r-project.org}{www.r-project.org}
\item
  The download site is called CRAN --- the \textbf{Comprehensive R
  Archive Network}
\item
  CRAN is a series of mirrored web servers to spread the load of
  thousands of users downloading R and associated packages
\item
  The CRAN master is at:
  \href{http://cran.r-project.org}{cran.r-project.org}
\end{itemize}

\end{frame}

\begin{frame}[fragile]{Starting R and other preliminaries}

\begin{itemize}
\tightlist
\item
  You start R in a variety of ways depending on your OS
\item
  R starts in a \textbf{working directory} where it looks for files and
  saves objects
\item
  Best to run R in a new directory for each project or analysis task
\item
  \texttt{getwd()} and \texttt{setwd()} get and set the working
  directory
\item
  To exit R, the function \texttt{q()} is used
\item
  You will be asked if you want to save your workspace; invariably you
  should answer \texttt{n} to this
\end{itemize}

\end{frame}

\begin{frame}[fragile]{Getting help}

\begin{itemize}
\tightlist
\item
  R comes with a lot of documentation
\item
  To get help on functions or concepts within R, use the \texttt{"?"}
  operator
\item
  For help on the \texttt{getwd()} function use: \texttt{?getwd}
\item
  Function \texttt{help.search("foo")} will search through all packages
  installed for help pages with \texttt{"foo"} in them
\item
  How the help is displayed is system dependent
\item
  Google is your friend
\item
  StackOverflow's R tag:
  \href{http://stackoverflow.com/questions/tagged/r}{stackoverflow.com/questions/tagged/r}
\end{itemize}

\end{frame}

\begin{frame}[fragile]{Working with R \& entering commands}

\begin{itemize}
\tightlist
\item
  Type commands at prompt \texttt{"\textgreater{}"} and these are
  evaluated when you hit RETURN
\item
  If a line is not syntactically complete, the prompt is changed to
  \texttt{"+"}
\item
  If returned object not assigned, it is printed to console
\item
  Assigning the results of a function call achieved by the assignment
  operator \texttt{"\textless{}-"}
\item
  Whatever is on the right of \texttt{"\textless{}-"} is assigned to the
  object named on the left of \texttt{"\textless{}-"}
\item
  Enter the name of an object and hit RETURN to print the contents
\item
  \texttt{ls()} returns a list of objects currently in your workspace
\end{itemize}

\end{frame}

\begin{frame}[fragile]{Working with R \& entering commands}

\begin{Shaded}
\begin{Highlighting}[]
\NormalTok{>}\StringTok{ }\DecValTok{5} \NormalTok{*}\StringTok{ }\DecValTok{3}
\end{Highlighting}
\end{Shaded}

\begin{verbatim}
[1] 15
\end{verbatim}

\begin{Shaded}
\begin{Highlighting}[]
\NormalTok{>}\StringTok{ }\NormalTok{radius <-}\StringTok{ }\DecValTok{5}
\NormalTok{>}\StringTok{ }\NormalTok{pi *}\StringTok{ }\NormalTok{radius^}\DecValTok{2}
\end{Highlighting}
\end{Shaded}

\begin{verbatim}
[1] 78.53982
\end{verbatim}

\begin{Shaded}
\begin{Highlighting}[]
\NormalTok{>}\StringTok{ }\NormalTok{ans <-}\StringTok{ }\DecValTok{5} \NormalTok{*}\StringTok{ }\DecValTok{3}
\NormalTok{>}\StringTok{ }\NormalTok{ans}
\end{Highlighting}
\end{Shaded}

\begin{verbatim}
[1] 15
\end{verbatim}

\begin{Shaded}
\begin{Highlighting}[]
\NormalTok{>}\StringTok{ }\NormalTok{ans2 <-}\StringTok{ }\NormalTok{ans +}\StringTok{ }\DecValTok{20}
\NormalTok{>}\StringTok{ }\NormalTok{ans2}
\end{Highlighting}
\end{Shaded}

\begin{verbatim}
[1] 35
\end{verbatim}

\begin{Shaded}
\begin{Highlighting}[]
\NormalTok{>}\StringTok{ }\KeywordTok{ls}\NormalTok{()}
\end{Highlighting}
\end{Shaded}

\begin{verbatim}
[1] "ans"    "ans2"   "radius"
\end{verbatim}

\end{frame}

\section{Data structures}\label{data-structures}

\begin{frame}[fragile]{The basic data structures}

Following Hadley Wickham's \emph{Advanced R}, the basic data structures
in R can be classified according to

\begin{enumerate}
\def\labelenumi{\arabic{enumi}.}
\item
  whether their contents are all the same (homogeneous) or not,
\item
  the dimensionality of the object

  \begin{longtable}[]{@{}rll@{}}
  \toprule
  Homo & geneous Hete & rogeneous\tabularnewline
  \midrule
  \endhead
  1d & Atomic vector & List\tabularnewline
  2d & Matrix & Data frame\tabularnewline
  nd & Array & NA\tabularnewline
  \bottomrule
  \end{longtable}
\end{enumerate}

The best way to understand which data structure an R object is, is to
use \texttt{str()}

\end{frame}

\begin{frame}[fragile]{Vectors}

Vectors are the basic type of data object in R and there are two main
types

\begin{enumerate}
\def\labelenumi{\arabic{enumi}.}
\tightlist
\item
  \emph{atomic} vectors
\item
  lists
\end{enumerate}

Each with three common properties

\begin{itemize}
\tightlist
\item
  What type of vector the object is: \texttt{typeof()}
\item
  Its length: \texttt{length()}
\item
  Attributes, which are extra information or metadata:
  \texttt{attributes()}, \texttt{attr()}
\end{itemize}

\end{frame}

\begin{frame}{Atomic vectors}

\textbf{Atomic} vectors differ fundamentally from lists because atomic
vectors can only contain elements that are \emph{all of the same type}

The four main types of atomic vector are

\begin{enumerate}
\def\labelenumi{\arabic{enumi}.}
\tightlist
\item
  logical
\item
  integer
\item
  double
\item
  character
\end{enumerate}

Two other types are less often encountered; raw and complex vectors

\end{frame}

\begin{frame}[fragile]{Atomic vectors}

We create atomic vector with \texttt{c()}, the \emph{concatenate} or
\emph{combine} function

\begin{Shaded}
\begin{Highlighting}[]
\NormalTok{>}\StringTok{ }\NormalTok{dbl <-}\StringTok{ }\KeywordTok{c}\NormalTok{(}\DecValTok{1}\NormalTok{, }\DecValTok{2}\NormalTok{, }\DecValTok{3}\NormalTok{)}
\NormalTok{>}\StringTok{ }\NormalTok{int <-}\StringTok{ }\KeywordTok{c}\NormalTok{(1L, 2L, 3L)}
\NormalTok{>}\StringTok{ }\NormalTok{logi <-}\StringTok{ }\KeywordTok{c}\NormalTok{(}\OtherTok{TRUE}\NormalTok{, }\OtherTok{FALSE}\NormalTok{, }\OtherTok{TRUE}\NormalTok{)            }\CommentTok{# Avoid using c(T, F, T)}
\NormalTok{>}\StringTok{ }\NormalTok{chr <-}\StringTok{ }\KeywordTok{c}\NormalTok{(}\StringTok{"Hello"}\NormalTok{, }\StringTok{"World"}\NormalTok{)}
\end{Highlighting}
\end{Shaded}

If an element (observation) is missing, use \texttt{NA}

\end{frame}

\begin{frame}[fragile]{Atomic vectors --- coercion}

All elements of atomic vectors must be of the same type. If you mix
types or attempt to combine atomic vectors of different types, they will
be \emph{coerced} towards the most general type.

The ordering is (from least to most general)

\begin{enumerate}
\def\labelenumi{\arabic{enumi}.}
\tightlist
\item
  logical
\item
  integer
\item
  double
\item
  character
\end{enumerate}

\begin{Shaded}
\begin{Highlighting}[]
\NormalTok{>}\StringTok{ }\KeywordTok{str}\NormalTok{(}\KeywordTok{c}\NormalTok{(}\StringTok{"a"}\NormalTok{, }\DecValTok{1}\NormalTok{))}
\end{Highlighting}
\end{Shaded}

\begin{verbatim}
 chr [1:2] "a" "1"
\end{verbatim}

\begin{Shaded}
\begin{Highlighting}[]
\NormalTok{>}\StringTok{ }\KeywordTok{str}\NormalTok{(}\KeywordTok{c}\NormalTok{(}\OtherTok{TRUE}\NormalTok{, }\DecValTok{10}\NormalTok{))}
\end{Highlighting}
\end{Shaded}

\begin{verbatim}
 num [1:2] 1 10
\end{verbatim}

\begin{Shaded}
\begin{Highlighting}[]
\NormalTok{>}\StringTok{ }\KeywordTok{str}\NormalTok{(}\KeywordTok{c}\NormalTok{(10L, }\DecValTok{2}\NormalTok{))}
\end{Highlighting}
\end{Shaded}

\begin{verbatim}
 num [1:2] 10 2
\end{verbatim}

\end{frame}

\begin{frame}[fragile]{Atomic vectors --- coercion}

One handy coercion is the conversion of logical vectors to numeric
(integer or double).

A \texttt{TRUE} is \texttt{1}, whilst a \texttt{FALSE} is \texttt{0},
which allows them to be used in numeric mathematical operations

Coercion happens atomically, but you can control this by explicitly
coercing to the required type with

\begin{itemize}
\tightlist
\item
  \texttt{as.characer()}
\item
  \texttt{as.double()}
\item
  \texttt{as.integer()}
\item
  \texttt{as.logical()}
\end{itemize}

\begin{Shaded}
\begin{Highlighting}[]
\NormalTok{>}\StringTok{ }\NormalTok{x <-}\StringTok{ }\KeywordTok{c}\NormalTok{(}\OtherTok{FALSE}\NormalTok{, }\OtherTok{FALSE}\NormalTok{, }\OtherTok{TRUE}\NormalTok{)}
\NormalTok{>}\StringTok{ }\KeywordTok{as.numeric}\NormalTok{(x)}
\end{Highlighting}
\end{Shaded}

\begin{verbatim}
[1] 0 0 1
\end{verbatim}

\begin{Shaded}
\begin{Highlighting}[]
\NormalTok{>}\StringTok{ }\KeywordTok{sum}\NormalTok{(x)                                  }\CommentTok{# count up the TRUEs}
\end{Highlighting}
\end{Shaded}

\begin{verbatim}
[1] 1
\end{verbatim}

\begin{Shaded}
\begin{Highlighting}[]
\NormalTok{>}\StringTok{ }\KeywordTok{mean}\NormalTok{(x)                                 }\CommentTok{# proportion of TRUE}
\end{Highlighting}
\end{Shaded}

\begin{verbatim}
[1] 0.3333333
\end{verbatim}

\end{frame}

\begin{frame}[fragile]{Lists}

Lists are exceedingly common in R --- it's often how fitted statistical
models are stored

Lists are \emph{general} vectors because they elements they contain can
be of any type --- lists can even contain lists

\begin{Shaded}
\begin{Highlighting}[]
\NormalTok{>}\StringTok{ }\NormalTok{x <-}\StringTok{ }\KeywordTok{list}\NormalTok{(}\DecValTok{1}\NormalTok{:}\DecValTok{3}\NormalTok{, }\StringTok{"a"}\NormalTok{, }\KeywordTok{c}\NormalTok{(}\OtherTok{TRUE}\NormalTok{, }\OtherTok{FALSE}\NormalTok{, }\OtherTok{TRUE}\NormalTok{), }\KeywordTok{c}\NormalTok{(}\FloatTok{2.3}\NormalTok{, }\FloatTok{5.9}\NormalTok{))}
\NormalTok{>}\StringTok{ }\KeywordTok{str}\NormalTok{(x)}
\end{Highlighting}
\end{Shaded}

\begin{verbatim}
List of 4
 $ : int [1:3] 1 2 3
 $ : chr "a"
 $ : logi [1:3] TRUE FALSE TRUE
 $ : num [1:2] 2.3 5.9
\end{verbatim}

\begin{Shaded}
\begin{Highlighting}[]
\NormalTok{>}\StringTok{ }\NormalTok{x <-}\StringTok{ }\KeywordTok{list}\NormalTok{(}\KeywordTok{list}\NormalTok{(}\KeywordTok{list}\NormalTok{(list)))}
\NormalTok{>}\StringTok{ }\KeywordTok{str}\NormalTok{(x)}
\end{Highlighting}
\end{Shaded}

\begin{verbatim}
List of 1
 $ :List of 1
  ..$ :List of 1
  .. ..$ :function (...)  
\end{verbatim}

\end{frame}

\begin{frame}[fragile]{Factors}

Factors are useful for storing data where observations can take on one
of a finite set of values

Factors combine integer vectors with attributes to create a new data
structure

They have \texttt{class()} \texttt{"factor"} and a \texttt{levels()}
attribute which lists the set of values the elements can take

\begin{Shaded}
\begin{Highlighting}[]
\NormalTok{>}\StringTok{ }\NormalTok{x <-}\StringTok{ }\KeywordTok{factor}\NormalTok{(}\KeywordTok{c}\NormalTok{(}\StringTok{"x"}\NormalTok{,}\StringTok{"y"}\NormalTok{,}\StringTok{"z"}\NormalTok{,}\StringTok{"z"}\NormalTok{,}\StringTok{"x"}\NormalTok{,}\StringTok{"z"}\NormalTok{,}\StringTok{"y"}\NormalTok{))}
\NormalTok{>}\StringTok{ }\NormalTok{x}
\end{Highlighting}
\end{Shaded}

\begin{verbatim}
[1] x y z z x z y
Levels: x y z
\end{verbatim}

\begin{Shaded}
\begin{Highlighting}[]
\NormalTok{>}\StringTok{ }\KeywordTok{class}\NormalTok{(x)}
\end{Highlighting}
\end{Shaded}

\begin{verbatim}
[1] "factor"
\end{verbatim}

\begin{Shaded}
\begin{Highlighting}[]
\NormalTok{>}\StringTok{ }\KeywordTok{levels}\NormalTok{(x)}
\end{Highlighting}
\end{Shaded}

\begin{verbatim}
[1] "x" "y" "z"
\end{verbatim}

\begin{Shaded}
\begin{Highlighting}[]
\NormalTok{>}\StringTok{ }\NormalTok{x[}\DecValTok{2}\NormalTok{] <-}\StringTok{ "c"}                             \CommentTok{# throws a warning}
\end{Highlighting}
\end{Shaded}

\begin{verbatim}
Warning in `[<-.factor`(`*tmp*`, 2, value = "c"): invalid factor level, NA
generated
\end{verbatim}

\begin{Shaded}
\begin{Highlighting}[]
\NormalTok{>}\StringTok{ }\NormalTok{x}
\end{Highlighting}
\end{Shaded}

\begin{verbatim}
[1] x    <NA> z    z    x    z    y   
Levels: x y z
\end{verbatim}

\begin{Shaded}
\begin{Highlighting}[]
\NormalTok{>}\StringTok{ }\NormalTok{x <-}\StringTok{ }\KeywordTok{factor}\NormalTok{(}\KeywordTok{c}\NormalTok{(}\StringTok{"x"}\NormalTok{,}\StringTok{"y"}\NormalTok{,}\StringTok{"z"}\NormalTok{,}\StringTok{"z"}\NormalTok{), }\DataTypeTok{levels =} \KeywordTok{c}\NormalTok{(}\StringTok{"z"}\NormalTok{,}\StringTok{"y"}\NormalTok{,}\StringTok{"x"}\NormalTok{)) }\CommentTok{# specify levels}
\end{Highlighting}
\end{Shaded}

\end{frame}

\begin{frame}[fragile]{Matrices \& arrays}

What separates atomic vectors from arrays is the presence of a
\texttt{dim()} attribute

As arrays are really atomic vectors with this extra attribute, they can
only contain elements of a single type

Matrices are a special case of a multi-dimensional array, where there
are only two dimensions

\begin{Shaded}
\begin{Highlighting}[]
\NormalTok{>}\StringTok{ }\NormalTok{m <-}\StringTok{ }\KeywordTok{matrix}\NormalTok{(}\DecValTok{1}\NormalTok{:}\DecValTok{6}\NormalTok{, }\DataTypeTok{ncol =} \DecValTok{3}\NormalTok{, }\DataTypeTok{nrow =} \DecValTok{2}\NormalTok{)}
\NormalTok{>}\StringTok{ }\NormalTok{m}
\end{Highlighting}
\end{Shaded}

\begin{verbatim}
     [,1] [,2] [,3]
[1,]    1    3    5
[2,]    2    4    6
\end{verbatim}

\begin{Shaded}
\begin{Highlighting}[]
\NormalTok{>}\StringTok{ }\NormalTok{m <-}\StringTok{ }\KeywordTok{matrix}\NormalTok{(}\DecValTok{1}\NormalTok{:}\DecValTok{6}\NormalTok{, }\KeywordTok{c}\NormalTok{(}\DecValTok{3}\NormalTok{,}\DecValTok{2}\NormalTok{))}
\NormalTok{>}\StringTok{ }
\ErrorTok{>}\StringTok{ }\NormalTok{m <-}\StringTok{ }\DecValTok{1}\NormalTok{:}\DecValTok{6}
\NormalTok{>}\StringTok{ }\KeywordTok{dim}\NormalTok{(m) <-}\StringTok{ }\KeywordTok{c}\NormalTok{(}\DecValTok{3}\NormalTok{,}\DecValTok{2}\NormalTok{)}
\NormalTok{>}\StringTok{ }\KeywordTok{dim}\NormalTok{(m) <-}\StringTok{ }\KeywordTok{c}\NormalTok{(}\DecValTok{2}\NormalTok{,}\DecValTok{3}\NormalTok{)}
\NormalTok{>}\StringTok{ }\NormalTok{m}
\end{Highlighting}
\end{Shaded}

\begin{verbatim}
     [,1] [,2] [,3]
[1,]    1    3    5
[2,]    2    4    6
\end{verbatim}

\end{frame}

\begin{frame}[fragile]{Data frames}

Data frames are a bit like Excel worksheets for R; they are like 2-d
matrices with the exception that the columns can store different types
of objects

Internally, data frames are lists, with the extra restriction that each
component (column) of the data frame has to be of the same length

Data frame can be created using \texttt{data.frame()}

\begin{Shaded}
\begin{Highlighting}[]
\NormalTok{>}\StringTok{ }\NormalTok{df <-}\StringTok{ }\KeywordTok{data.frame}\NormalTok{(}\DataTypeTok{x =} \DecValTok{1}\NormalTok{:}\DecValTok{3}\NormalTok{, }\DataTypeTok{y =} \KeywordTok{c}\NormalTok{(}\StringTok{"a"}\NormalTok{, }\StringTok{"b"}\NormalTok{, }\StringTok{"c"}\NormalTok{))}
\NormalTok{>}\StringTok{ }\KeywordTok{str}\NormalTok{(df)}
\end{Highlighting}
\end{Shaded}

\begin{verbatim}
'data.frame':   3 obs. of  2 variables:
 $ x: int  1 2 3
 $ y: Factor w/ 3 levels "a","b","c": 1 2 3
\end{verbatim}

\begin{Shaded}
\begin{Highlighting}[]
\NormalTok{>}\StringTok{ }\NormalTok{df <-}\StringTok{ }\KeywordTok{data.frame}\NormalTok{(}\DataTypeTok{x =} \DecValTok{1}\NormalTok{:}\DecValTok{3}\NormalTok{, }\DataTypeTok{y =} \KeywordTok{c}\NormalTok{(}\StringTok{"a"}\NormalTok{, }\StringTok{"b"}\NormalTok{, }\StringTok{"c"}\NormalTok{), }\DataTypeTok{stringsAsFactors =} \OtherTok{FALSE}\NormalTok{)}
\NormalTok{>}\StringTok{ }\KeywordTok{is.data.frame}\NormalTok{(df)}
\end{Highlighting}
\end{Shaded}

\begin{verbatim}
[1] TRUE
\end{verbatim}

\begin{Shaded}
\begin{Highlighting}[]
\NormalTok{>}\StringTok{ }\KeywordTok{nrow}\NormalTok{(df)}
\end{Highlighting}
\end{Shaded}

\begin{verbatim}
[1] 3
\end{verbatim}

\begin{Shaded}
\begin{Highlighting}[]
\NormalTok{>}\StringTok{ }\KeywordTok{class}\NormalTok{(df)}
\end{Highlighting}
\end{Shaded}

\begin{verbatim}
[1] "data.frame"
\end{verbatim}

\end{frame}

\begin{frame}[fragile]{Data frames II}

Additional columns and rows can be added to a data frame using
\texttt{cbind()} and \texttt{rbind()} respectively

\begin{Shaded}
\begin{Highlighting}[]
\NormalTok{>}\StringTok{ }\KeywordTok{cbind}\NormalTok{(df, }\KeywordTok{data.frame}\NormalTok{(}\DataTypeTok{z =} \DecValTok{3}\NormalTok{:}\DecValTok{1}\NormalTok{))}
\end{Highlighting}
\end{Shaded}

\begin{verbatim}
  x y z
1 1 a 3
2 2 b 2
3 3 c 1
\end{verbatim}

\begin{Shaded}
\begin{Highlighting}[]
\NormalTok{>}\StringTok{ }\KeywordTok{rbind}\NormalTok{(df, }\KeywordTok{data.frame}\NormalTok{(}\DataTypeTok{x =} \DecValTok{10}\NormalTok{, }\DataTypeTok{y =} \StringTok{"z"}\NormalTok{))}
\end{Highlighting}
\end{Shaded}

\begin{verbatim}
   x y
1  1 a
2  2 b
3  3 c
4 10 z
\end{verbatim}

\end{frame}

\begin{frame}{Subsetting}

Subsetting is an incredibly useful and powerful way of manipulating data
objects --- but it is hard to learn initially as there is a lot of
detail to remember

\end{frame}

\begin{frame}[fragile]{Subsetting atomic vectors}

\columnsbegin
\column{0.7\linewidth}

The main subsetting function is \texttt{{[}} --- it takes one of several
types of arguments which determines how the vector is subset

\begin{itemize}
\tightlist
\item
  \emph{positive integers} return the specified elements
\item
  \emph{negative integers} omit the specified elements from the result
\item
  \emph{logical vectors} return elements where the logical vector is
  \texttt{TRUE}
\item
  \emph{nothing} returns the original vector
\item
  \emph{zero} returns a zero-length vector
\end{itemize}

Subsetting a list works the same way with \texttt{{[}}

\column{0.3\linewidth}

\begin{Shaded}
\begin{Highlighting}[]
\NormalTok{>}\StringTok{ }\NormalTok{x <-}\StringTok{ }\KeywordTok{c}\NormalTok{(}\FloatTok{1.3}\NormalTok{, }\FloatTok{4.5}\NormalTok{, }\FloatTok{2.3}\NormalTok{, }\FloatTok{4.2}\NormalTok{, }\FloatTok{5.4}\NormalTok{)}
\NormalTok{>}\StringTok{ }\NormalTok{x[}\KeywordTok{c}\NormalTok{(}\DecValTok{3}\NormalTok{, }\DecValTok{1}\NormalTok{)]}
\end{Highlighting}
\end{Shaded}

\begin{verbatim}
[1] 2.3 1.3
\end{verbatim}

\begin{Shaded}
\begin{Highlighting}[]
\NormalTok{>}\StringTok{ }\NormalTok{x[-}\KeywordTok{c}\NormalTok{(}\DecValTok{3}\NormalTok{, }\DecValTok{1}\NormalTok{)]}
\end{Highlighting}
\end{Shaded}

\begin{verbatim}
[1] 4.5 4.2 5.4
\end{verbatim}

\begin{Shaded}
\begin{Highlighting}[]
\NormalTok{>}\StringTok{ }\NormalTok{x[x >}\StringTok{ }\DecValTok{3}\NormalTok{]}
\end{Highlighting}
\end{Shaded}

\begin{verbatim}
[1] 4.5 4.2 5.4
\end{verbatim}

\begin{Shaded}
\begin{Highlighting}[]
\NormalTok{>}\StringTok{ }\NormalTok{x[}\KeywordTok{c}\NormalTok{(}\OtherTok{TRUE}\NormalTok{, }\OtherTok{FALSE}\NormalTok{)]}
\end{Highlighting}
\end{Shaded}

\begin{verbatim}
[1] 1.3 2.3 5.4
\end{verbatim}

\begin{Shaded}
\begin{Highlighting}[]
\NormalTok{>}\StringTok{ }\NormalTok{x[]}
\end{Highlighting}
\end{Shaded}

\begin{verbatim}
[1] 1.3 4.5 2.3 4.2 5.4
\end{verbatim}

\begin{Shaded}
\begin{Highlighting}[]
\NormalTok{>}\StringTok{ }\NormalTok{x[}\DecValTok{0}\NormalTok{]}
\end{Highlighting}
\end{Shaded}

\begin{verbatim}
numeric(0)
\end{verbatim}

\begin{Shaded}
\begin{Highlighting}[]
\NormalTok{>}\StringTok{ }\NormalTok{(y <-}\StringTok{ }\KeywordTok{setNames}\NormalTok{(x, letters[}\DecValTok{1}\NormalTok{:}\DecValTok{5}\NormalTok{]))}
\end{Highlighting}
\end{Shaded}

\begin{verbatim}
  a   b   c   d   e 
1.3 4.5 2.3 4.2 5.4 
\end{verbatim}

\begin{Shaded}
\begin{Highlighting}[]
\NormalTok{>}\StringTok{ }\NormalTok{y[}\KeywordTok{c}\NormalTok{(}\StringTok{"a"}\NormalTok{, }\StringTok{"c"}\NormalTok{)]}
\end{Highlighting}
\end{Shaded}

\begin{verbatim}
  a   c 
1.3 2.3 
\end{verbatim}

\columnsend

\end{frame}

\begin{frame}[fragile]{Subsetting matrices and arrays}

\columnsbegin
\column{0.7\linewidth}

The main subsetting function is \texttt{{[}} --- it takes one of several
types of arguments which determines how the array is subset

\begin{itemize}
\tightlist
\item
  a pair of 1-d vectors, one for each dimension
\item
  a single 1-d vector
\item
  a matrix
\end{itemize}

\column{0.3\linewidth}

\begin{Shaded}
\begin{Highlighting}[]
\NormalTok{>}\StringTok{ }\NormalTok{a <-}\StringTok{ }\KeywordTok{matrix}\NormalTok{(}\DecValTok{1}\NormalTok{:}\DecValTok{9}\NormalTok{, }\DataTypeTok{nrow =} \DecValTok{3}\NormalTok{)}
\NormalTok{>}\StringTok{ }\KeywordTok{colnames}\NormalTok{(a) <-}\StringTok{ }\NormalTok{LETTERS[}\DecValTok{1}\NormalTok{:}\DecValTok{3}\NormalTok{]}
\NormalTok{>}\StringTok{ }\NormalTok{a[}\DecValTok{1}\NormalTok{:}\DecValTok{2}\NormalTok{, ]}
\end{Highlighting}
\end{Shaded}

\begin{verbatim}
     A B C
[1,] 1 4 7
[2,] 2 5 8
\end{verbatim}

\begin{Shaded}
\begin{Highlighting}[]
\NormalTok{>}\StringTok{ }\NormalTok{a[, }\DecValTok{2}\NormalTok{]}
\end{Highlighting}
\end{Shaded}

\begin{verbatim}
[1] 4 5 6
\end{verbatim}

\columnsend

\end{frame}

\begin{frame}[fragile]{Subsetting data frames}

\columnsbegin
\column{0.5\linewidth}

Again, the main subsetting function is \texttt{{[}} --- you can use it
to subset a data frame as if it were

\begin{itemize}
\tightlist
\item
  a matrix, with an indexing vector for the rows and the columns
\item
  a list
\end{itemize}

\column{0.5\linewidth}

\begin{Shaded}
\begin{Highlighting}[]
\NormalTok{>}\StringTok{ }\NormalTok{df <-}\StringTok{ }\KeywordTok{data.frame}\NormalTok{(}\DataTypeTok{x =} \DecValTok{1}\NormalTok{:}\DecValTok{3}\NormalTok{, }\DataTypeTok{y =} \DecValTok{3}\NormalTok{:}\DecValTok{1}\NormalTok{, }\DataTypeTok{z =} \NormalTok{letters[}\DecValTok{1}\NormalTok{:}\DecValTok{3}\NormalTok{])}
\NormalTok{>}\StringTok{ }\NormalTok{df[df$x ==}\StringTok{ }\DecValTok{2}\NormalTok{, ]}
\end{Highlighting}
\end{Shaded}

\begin{verbatim}
  x y z
2 2 2 b
\end{verbatim}

\begin{Shaded}
\begin{Highlighting}[]
\NormalTok{>}\StringTok{ }\NormalTok{df[}\KeywordTok{c}\NormalTok{(}\StringTok{"x"}\NormalTok{, }\StringTok{"y"}\NormalTok{)]}
\end{Highlighting}
\end{Shaded}

\begin{verbatim}
  x y
1 1 3
2 2 2
3 3 1
\end{verbatim}

\begin{Shaded}
\begin{Highlighting}[]
\NormalTok{>}\StringTok{ }\NormalTok{df[, }\KeywordTok{c}\NormalTok{(}\StringTok{"x"}\NormalTok{, }\StringTok{"y"}\NormalTok{)]}
\end{Highlighting}
\end{Shaded}

\begin{verbatim}
  x y
1 1 3
2 2 2
3 3 1
\end{verbatim}

\begin{Shaded}
\begin{Highlighting}[]
\NormalTok{>}\StringTok{ }\KeywordTok{str}\NormalTok{(df[}\StringTok{"x"}\NormalTok{])}
\end{Highlighting}
\end{Shaded}

\begin{verbatim}
'data.frame':   3 obs. of  1 variable:
 $ x: int  1 2 3
\end{verbatim}

\begin{Shaded}
\begin{Highlighting}[]
\NormalTok{>}\StringTok{ }\KeywordTok{str}\NormalTok{(df[, }\StringTok{"x"}\NormalTok{])                          }\CommentTok{# drops the enpty dimension}
\end{Highlighting}
\end{Shaded}

\begin{verbatim}
 int [1:3] 1 2 3
\end{verbatim}

\columnsend

\end{frame}

\begin{frame}{Re-use}

Copyright © (2015--2017) Gavin L. Simpson Some Rights Reserved

Unless indicated otherwise, this slide deck is licensed under a
\href{http://creativecommons.org/licenses/by/4.0/}{Creative Commons
Attribution 4.0 International License}.

\begin{center}
  \ccby
\end{center}

\end{frame}

\end{document}
