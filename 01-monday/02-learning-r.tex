\documentclass[10pt,ignorenonframetext,compress, aspectratio=169]{beamer}
\setbeamertemplate{caption}[numbered]
\setbeamertemplate{caption label separator}{: }
\setbeamercolor{caption name}{fg=normal text.fg}
\beamertemplatenavigationsymbolsempty
\usepackage{lmodern}
\usepackage{amssymb,amsmath,mathtools}
\usepackage{ifxetex,ifluatex}
\usepackage{fixltx2e} % provides \textsubscript
\ifnum 0\ifxetex 1\fi\ifluatex 1\fi=0 % if pdftex
  \usepackage[T1]{fontenc}
  \usepackage[utf8]{inputenc}
\else % if luatex or xelatex
  \ifxetex
    \usepackage{mathspec}
  \else
    \usepackage{fontspec}
  \fi
  %%\defaultfontfeatures{Ligatures=TeX,Scale=MatchLowercase}
  \defaultfontfeatures{Scale=MatchLowercase}
\fi
\usetheme[]{metropolis}
% use upquote if available, for straight quotes in verbatim environments
\IfFileExists{upquote.sty}{\usepackage{upquote}}{}
% use microtype if available
\IfFileExists{microtype.sty}{%
\usepackage{microtype}
\UseMicrotypeSet[protrusion]{basicmath} % disable protrusion for tt fonts
}{}
\newif\ifbibliography
\usepackage{color}
\usepackage{fancyvrb}
\newcommand{\VerbBar}{|}
\newcommand{\VERB}{\Verb[commandchars=\\\{\}]}
\DefineVerbatimEnvironment{Highlighting}{Verbatim}{commandchars=\\\{\}}
% Add ',fontsize=\small' for more characters per line
\usepackage{framed}
\definecolor{shadecolor}{RGB}{248,248,248}
\newenvironment{Shaded}{\begin{snugshade}}{\end{snugshade}}
\newcommand{\KeywordTok}[1]{\textcolor[rgb]{0.13,0.29,0.53}{\textbf{{#1}}}}
\newcommand{\DataTypeTok}[1]{\textcolor[rgb]{0.13,0.29,0.53}{{#1}}}
\newcommand{\DecValTok}[1]{\textcolor[rgb]{0.00,0.00,0.81}{{#1}}}
\newcommand{\BaseNTok}[1]{\textcolor[rgb]{0.00,0.00,0.81}{{#1}}}
\newcommand{\FloatTok}[1]{\textcolor[rgb]{0.00,0.00,0.81}{{#1}}}
\newcommand{\ConstantTok}[1]{\textcolor[rgb]{0.00,0.00,0.00}{{#1}}}
\newcommand{\CharTok}[1]{\textcolor[rgb]{0.31,0.60,0.02}{{#1}}}
\newcommand{\SpecialCharTok}[1]{\textcolor[rgb]{0.00,0.00,0.00}{{#1}}}
\newcommand{\StringTok}[1]{\textcolor[rgb]{0.31,0.60,0.02}{{#1}}}
\newcommand{\VerbatimStringTok}[1]{\textcolor[rgb]{0.31,0.60,0.02}{{#1}}}
\newcommand{\SpecialStringTok}[1]{\textcolor[rgb]{0.31,0.60,0.02}{{#1}}}
\newcommand{\ImportTok}[1]{{#1}}
\newcommand{\CommentTok}[1]{\textcolor[rgb]{0.56,0.35,0.01}{\textit{{#1}}}}
\newcommand{\DocumentationTok}[1]{\textcolor[rgb]{0.56,0.35,0.01}{\textbf{\textit{{#1}}}}}
\newcommand{\AnnotationTok}[1]{\textcolor[rgb]{0.56,0.35,0.01}{\textbf{\textit{{#1}}}}}
\newcommand{\CommentVarTok}[1]{\textcolor[rgb]{0.56,0.35,0.01}{\textbf{\textit{{#1}}}}}
\newcommand{\OtherTok}[1]{\textcolor[rgb]{0.56,0.35,0.01}{{#1}}}
\newcommand{\FunctionTok}[1]{\textcolor[rgb]{0.00,0.00,0.00}{{#1}}}
\newcommand{\VariableTok}[1]{\textcolor[rgb]{0.00,0.00,0.00}{{#1}}}
\newcommand{\ControlFlowTok}[1]{\textcolor[rgb]{0.13,0.29,0.53}{\textbf{{#1}}}}
\newcommand{\OperatorTok}[1]{\textcolor[rgb]{0.81,0.36,0.00}{\textbf{{#1}}}}
\newcommand{\BuiltInTok}[1]{{#1}}
\newcommand{\ExtensionTok}[1]{{#1}}
\newcommand{\PreprocessorTok}[1]{\textcolor[rgb]{0.56,0.35,0.01}{\textit{{#1}}}}
\newcommand{\AttributeTok}[1]{\textcolor[rgb]{0.77,0.63,0.00}{{#1}}}
\newcommand{\RegionMarkerTok}[1]{{#1}}
\newcommand{\InformationTok}[1]{\textcolor[rgb]{0.56,0.35,0.01}{\textbf{\textit{{#1}}}}}
\newcommand{\WarningTok}[1]{\textcolor[rgb]{0.56,0.35,0.01}{\textbf{\textit{{#1}}}}}
\newcommand{\AlertTok}[1]{\textcolor[rgb]{0.94,0.16,0.16}{{#1}}}
\newcommand{\ErrorTok}[1]{\textcolor[rgb]{0.64,0.00,0.00}{\textbf{{#1}}}}
\newcommand{\NormalTok}[1]{{#1}}
\usepackage{longtable,booktabs}
\usepackage{caption}
% These lines are needed to make table captions work with longtable:
\makeatletter
\def\fnum@table{\tablename~\thetable}
\makeatother

% Prevent slide breaks in the middle of a paragraph:
\widowpenalties 1 10000
\raggedbottom

\AtBeginPart{
  \let\insertpartnumber\relax
  \let\partname\relax
  \frame{\partpage}
}
\AtBeginSection{
  \ifbibliography
  \else
    \let\insertsectionnumber\relax
    \let\sectionname\relax
    \frame{\sectionpage}
  \fi
}
\AtBeginSubsection{
  \let\insertsubsectionnumber\relax
  \let\subsectionname\relax
  \frame{\subsectionpage}
}

\setlength{\parindent}{0pt}
\setlength{\parskip}{6pt plus 2pt minus 1pt}
\setlength{\emergencystretch}{3em}  % prevent overfull lines
\providecommand{\tightlist}{%
  \setlength{\itemsep}{0pt}\setlength{\parskip}{0pt}}
\setcounter{secnumdepth}{0}

%% GLS Added
% Textcomp for various common symbols
\usepackage{textcomp}

\usepackage{booktabs}

% Creative Commons Icons
\usepackage[scale=1]{ccicons}

\newenvironment{centrefig}{\begin{figure}\centering}{\end{figure}}
\newcommand{\columnsbegin}{\begin{columns}}
\newcommand{\columnsend}{\end{columns}}
\newcommand{\centreFigBegin}{\begin{figure}\centering}
\newcommand{\centreFigEnd}{\end{figure}}
%%

\DefineVerbatimEnvironment{Highlighting}{Verbatim}{commandchars=\\\{\}, fontsize=\tiny}
% make console-output smaller:
\makeatletter
\def\verbatim{\tiny\@verbatim \frenchspacing\@vobeyspaces \@xverbatim}
\makeatother
\setlength{\parskip}{0pt}
\setlength{\OuterFrameSep}{-4pt} % was -4pt
\makeatletter
\preto{\@verbatim}{\topsep=-10pt \partopsep=-10pt} % were -10pt
\makeatother

\title{Learning R}
\author{Gavin L. Simpson}
\date{February, 2017}

\begin{document}
\frame{\titlepage}

\section{Introduction to R}\label{introduction-to-r}

\begin{frame}{Why R?}

\begin{itemize}
\tightlist
\item
  R was designed from the ground up as a language for data analysis
\item
  It is free and open source, and available on all the major OSes
\item
  Huge package ecosystem (\textgreater{} 10,000) covering bewildering
  array of statistical methods, data visualizations, data import and
  manipulation
\item
  Cutting edge; R is used by thousands of statisticians \& R code or a
  package often accompanies papers developing new methods
\item
  For the most part, a great community providing help, blog posts etc
\end{itemize}

\end{frame}

\begin{frame}{What is R?}

\begin{itemize}
\tightlist
\item
  The S statistical language was started at Bell Labs on May 5, 1976
\item
  A system for general data analysis jobs that could replace the
  \emph{ad hoc} creation of FORTRAN applications
\item
  The S language was licensed by Insightful Corporation for use in their
  \emph{S-PLUS} software
\item
  In 2004 Insightful bought the S language from Lucent (formerly AT\&T
  and before that Bell Labs)
\item
  Robert Gentleman and Ross Ihaka designed a language that was
  compatible with S but which worked in a different way internally
\item
  They called this language R
\item
  There was a lot of interest in R and eventually it was made Open
  Source under the GNU GPL-2
\item
  R has drawn around it a group of dedicated stewards of the R software
  --- \alert{R Core}
\end{itemize}

\end{frame}

\begin{frame}{R on the web}

\begin{itemize}
\tightlist
\item
  The R homepage is located at:
  \href{http://www.r-project.org}{www.r-project.org}
\item
  The download site is called CRAN --- the \textbf{Comprehensive R
  Archive Network}
\item
  CRAN is a series of mirrored web servers to spread the load of
  thousands of users downloading R and associated packages
\item
  The CRAN master is at:
  \href{http://cran.r-project.org}{cran.r-project.org}
\end{itemize}

\end{frame}

\begin{frame}[fragile]{Starting R and other preliminaries}

\begin{itemize}
\tightlist
\item
  You start R in a variety of ways depending on your OS
\item
  R starts in a \textbf{working directory} where it looks for files and
  saves objects
\item
  Best to run R in a new directory for each project or analysis task
\item
  \texttt{getwd()} and \texttt{setwd()} get and set the working
  directory
\item
  To exit R, the function \texttt{q()} is used
\item
  You will be asked if you want to save your workspace; invariably you
  should answer \texttt{n} to this
\end{itemize}

\end{frame}

\begin{frame}[fragile]{Getting help}

\begin{itemize}
\tightlist
\item
  R comes with a lot of documentation
\item
  To get help on functions or concepts within R, use the \texttt{"?"}
  operator
\item
  For help on the \texttt{getwd()} function use: \texttt{?getwd}
\item
  Function \texttt{help.search("foo")} will search through all packages
  installed for help pages with \texttt{"foo"} in them
\item
  How the help is displayed is system dependent
\item
  Google is your friend
\item
  StackOverflow's R tag:
  \href{http://stackoverflow.com/questions/tagged/r}{stackoverflow.com/questions/tagged/r}
\end{itemize}

\end{frame}

\begin{frame}[fragile]{Working with R \& entering commands}

\begin{itemize}
\tightlist
\item
  Type commands at prompt \texttt{"\textgreater{}"} and these are
  evaluated when you hit RETURN
\item
  If a line is not syntactically complete, the prompt is changed to
  \texttt{"+"}
\item
  If returned object not assigned, it is printed to console
\item
  Assigning the results of a function call achieved by the assignment
  operator \texttt{"\textless{}-"}
\item
  Whatever is on the right of \texttt{"\textless{}-"} is assigned to the
  object named on the left of \texttt{"\textless{}-"}
\item
  Enter the name of an object and hit RETURN to print the contents
\item
  \texttt{ls()} returns a list of objects currently in your workspace
\end{itemize}

\end{frame}

\begin{frame}[fragile]{Working with R \& entering commands}

\begin{Shaded}
\begin{Highlighting}[]
\NormalTok{>}\StringTok{ }\DecValTok{5} \NormalTok{*}\StringTok{ }\DecValTok{3}
\end{Highlighting}
\end{Shaded}

\begin{verbatim}
[1] 15
\end{verbatim}

\begin{Shaded}
\begin{Highlighting}[]
\NormalTok{>}\StringTok{ }\NormalTok{radius <-}\StringTok{ }\DecValTok{5}
\NormalTok{>}\StringTok{ }\NormalTok{pi *}\StringTok{ }\NormalTok{radius^}\DecValTok{2}
\end{Highlighting}
\end{Shaded}

\begin{verbatim}
[1] 78.53982
\end{verbatim}

\begin{Shaded}
\begin{Highlighting}[]
\NormalTok{>}\StringTok{ }\NormalTok{ans <-}\StringTok{ }\DecValTok{5} \NormalTok{*}\StringTok{ }\DecValTok{3}
\NormalTok{>}\StringTok{ }\NormalTok{ans}
\end{Highlighting}
\end{Shaded}

\begin{verbatim}
[1] 15
\end{verbatim}

\begin{Shaded}
\begin{Highlighting}[]
\NormalTok{>}\StringTok{ }\NormalTok{ans2 <-}\StringTok{ }\NormalTok{ans +}\StringTok{ }\DecValTok{20}
\NormalTok{>}\StringTok{ }\NormalTok{ans2}
\end{Highlighting}
\end{Shaded}

\begin{verbatim}
[1] 35
\end{verbatim}

\begin{Shaded}
\begin{Highlighting}[]
\NormalTok{>}\StringTok{ }\KeywordTok{ls}\NormalTok{()}
\end{Highlighting}
\end{Shaded}

\begin{verbatim}
[1] "ans"    "ans2"   "radius"
\end{verbatim}

\end{frame}

\section{Data structures}\label{data-structures}

\begin{frame}[fragile]{The basic data structures}

Following Hadley Wickham's \emph{Advanced R}, the basic data structures
in R can be classified according to

\begin{enumerate}
\def\labelenumi{\arabic{enumi}.}
\item
  whether their contents are all the same (homogeneous) or not,
\item
  the dimensionality of the object

  \begin{longtable}[]{@{}rll@{}}
  \toprule
  Homo & geneous Hete & rogeneous\tabularnewline
  \midrule
  \endhead
  1d & Atomic vector & List\tabularnewline
  2d & Matrix & Data frame\tabularnewline
  nd & Array & NA\tabularnewline
  \bottomrule
  \end{longtable}
\end{enumerate}

The best way to understand which data structure an R object is, is to
use \texttt{str()}

\end{frame}

\begin{frame}[fragile]{Vectors}

Vectors are the basic type of data object in R and there are two main
types

\begin{enumerate}
\def\labelenumi{\arabic{enumi}.}
\tightlist
\item
  \emph{atomic} vectors
\item
  lists
\end{enumerate}

Each with three common properties

\begin{itemize}
\tightlist
\item
  What type of vector the object is: \texttt{typeof()}
\item
  Its length: \texttt{length()}
\item
  Attributes, which are extra information or metadata:
  \texttt{attributes()}, \texttt{attr()}
\end{itemize}

\end{frame}

\begin{frame}{Atomic vectors}

\textbf{Atomic} vectors differ fundamentally from lists because atomic
vectors can only contain elements that are \emph{all of the same type}

The four main types of atomic vector are

\begin{enumerate}
\def\labelenumi{\arabic{enumi}.}
\tightlist
\item
  logical
\item
  integer
\item
  double
\item
  character
\end{enumerate}

Two other types are less often encountered; raw and complex vectors

\end{frame}

\begin{frame}[fragile]{Atomic vectors}

We create atomic vector with \texttt{c()}, the \emph{concatenate} or
\emph{combine} function

\begin{Shaded}
\begin{Highlighting}[]
\NormalTok{>}\StringTok{ }\NormalTok{dbl <-}\StringTok{ }\KeywordTok{c}\NormalTok{(}\DecValTok{1}\NormalTok{, }\DecValTok{2}\NormalTok{, }\DecValTok{3}\NormalTok{)}
\NormalTok{>}\StringTok{ }\NormalTok{int <-}\StringTok{ }\KeywordTok{c}\NormalTok{(1L, 2L, 3L)}
\NormalTok{>}\StringTok{ }\NormalTok{logi <-}\StringTok{ }\KeywordTok{c}\NormalTok{(}\OtherTok{TRUE}\NormalTok{, }\OtherTok{FALSE}\NormalTok{, }\OtherTok{TRUE}\NormalTok{)            }\CommentTok{# Avoid using c(T, F, T)}
\NormalTok{>}\StringTok{ }\NormalTok{chr <-}\StringTok{ }\KeywordTok{c}\NormalTok{(}\StringTok{"Hello"}\NormalTok{, }\StringTok{"World"}\NormalTok{)}
\end{Highlighting}
\end{Shaded}

If an element (observation) is missing, use \texttt{NA}

\end{frame}

\begin{frame}[fragile]{Atomic vectors --- coercion}

All elements of atomic vectors must be of the same type. If you mix
types or attempt to combine atomic vectors of different types, they will
be \emph{coerced} towards the most general type.

The ordering is (from least to most general)

\begin{enumerate}
\def\labelenumi{\arabic{enumi}.}
\tightlist
\item
  logical
\item
  integer
\item
  double
\item
  character
\end{enumerate}

\begin{Shaded}
\begin{Highlighting}[]
\NormalTok{>}\StringTok{ }\KeywordTok{str}\NormalTok{(}\KeywordTok{c}\NormalTok{(}\StringTok{"a"}\NormalTok{, }\DecValTok{1}\NormalTok{))}
\end{Highlighting}
\end{Shaded}

\begin{verbatim}
 chr [1:2] "a" "1"
\end{verbatim}

\begin{Shaded}
\begin{Highlighting}[]
\NormalTok{>}\StringTok{ }\KeywordTok{str}\NormalTok{(}\KeywordTok{c}\NormalTok{(}\OtherTok{TRUE}\NormalTok{, }\DecValTok{10}\NormalTok{))}
\end{Highlighting}
\end{Shaded}

\begin{verbatim}
 num [1:2] 1 10
\end{verbatim}

\begin{Shaded}
\begin{Highlighting}[]
\NormalTok{>}\StringTok{ }\KeywordTok{str}\NormalTok{(}\KeywordTok{c}\NormalTok{(10L, }\DecValTok{2}\NormalTok{))}
\end{Highlighting}
\end{Shaded}

\begin{verbatim}
 num [1:2] 10 2
\end{verbatim}

\end{frame}

\begin{frame}[fragile]{Atomic vectors --- coercion}

One handy coercion is the conversion of logical vectors to numeric
(integer or double).

A \texttt{TRUE} is \texttt{1}, whilst a \texttt{FALSE} is \texttt{0},
which allows them to be used in numeric mathematical operations

Coercion happens atomically, but you can control this by explicitly
coercing to the required type with

\begin{itemize}
\tightlist
\item
  \texttt{as.characer()}
\item
  \texttt{as.double()}
\item
  \texttt{as.integer()}
\item
  \texttt{as.logical()}
\end{itemize}

\begin{Shaded}
\begin{Highlighting}[]
\NormalTok{>}\StringTok{ }\NormalTok{x <-}\StringTok{ }\KeywordTok{c}\NormalTok{(}\OtherTok{FALSE}\NormalTok{, }\OtherTok{FALSE}\NormalTok{, }\OtherTok{TRUE}\NormalTok{)}
\NormalTok{>}\StringTok{ }\KeywordTok{as.numeric}\NormalTok{(x)}
\end{Highlighting}
\end{Shaded}

\begin{verbatim}
[1] 0 0 1
\end{verbatim}

\begin{Shaded}
\begin{Highlighting}[]
\NormalTok{>}\StringTok{ }\KeywordTok{sum}\NormalTok{(x)                                  }\CommentTok{# count up the TRUEs}
\end{Highlighting}
\end{Shaded}

\begin{verbatim}
[1] 1
\end{verbatim}

\begin{Shaded}
\begin{Highlighting}[]
\NormalTok{>}\StringTok{ }\KeywordTok{mean}\NormalTok{(x)                                 }\CommentTok{# proportion of TRUE}
\end{Highlighting}
\end{Shaded}

\begin{verbatim}
[1] 0.3333333
\end{verbatim}

\end{frame}

\begin{frame}[fragile]{Lists}

Lists are exceedingly common in R --- it's often how fitted statistical
models are stored

Lists are \emph{general} vectors because they elements they contain can
be of any type --- lists can even contain lists

\begin{Shaded}
\begin{Highlighting}[]
\NormalTok{>}\StringTok{ }\NormalTok{x <-}\StringTok{ }\KeywordTok{list}\NormalTok{(}\DecValTok{1}\NormalTok{:}\DecValTok{3}\NormalTok{, }\StringTok{"a"}\NormalTok{, }\KeywordTok{c}\NormalTok{(}\OtherTok{TRUE}\NormalTok{, }\OtherTok{FALSE}\NormalTok{, }\OtherTok{TRUE}\NormalTok{), }\KeywordTok{c}\NormalTok{(}\FloatTok{2.3}\NormalTok{, }\FloatTok{5.9}\NormalTok{))}
\NormalTok{>}\StringTok{ }\KeywordTok{str}\NormalTok{(x)}
\end{Highlighting}
\end{Shaded}

\begin{verbatim}
List of 4
 $ : int [1:3] 1 2 3
 $ : chr "a"
 $ : logi [1:3] TRUE FALSE TRUE
 $ : num [1:2] 2.3 5.9
\end{verbatim}

\begin{Shaded}
\begin{Highlighting}[]
\NormalTok{>}\StringTok{ }\NormalTok{x <-}\StringTok{ }\KeywordTok{list}\NormalTok{(}\KeywordTok{list}\NormalTok{(}\KeywordTok{list}\NormalTok{(list)))}
\NormalTok{>}\StringTok{ }\KeywordTok{str}\NormalTok{(x)}
\end{Highlighting}
\end{Shaded}

\begin{verbatim}
List of 1
 $ :List of 1
  ..$ :List of 1
  .. ..$ :function (...)  
\end{verbatim}

\end{frame}

\begin{frame}[fragile]{Factors}

Factors are useful for storing data where observations can take on one
of a finite set of values

Factors combine integer vectors with attributes to create a new data
structure

They have \texttt{class()} \texttt{"factor"} and a \texttt{levels()}
attribute which lists the set of values the elements can take

\begin{Shaded}
\begin{Highlighting}[]
\NormalTok{>}\StringTok{ }\NormalTok{x <-}\StringTok{ }\KeywordTok{factor}\NormalTok{(}\KeywordTok{c}\NormalTok{(}\StringTok{"x"}\NormalTok{,}\StringTok{"y"}\NormalTok{,}\StringTok{"z"}\NormalTok{,}\StringTok{"z"}\NormalTok{,}\StringTok{"x"}\NormalTok{,}\StringTok{"z"}\NormalTok{,}\StringTok{"y"}\NormalTok{))}
\NormalTok{>}\StringTok{ }\NormalTok{x}
\end{Highlighting}
\end{Shaded}

\begin{verbatim}
[1] x y z z x z y
Levels: x y z
\end{verbatim}

\begin{Shaded}
\begin{Highlighting}[]
\NormalTok{>}\StringTok{ }\KeywordTok{class}\NormalTok{(x)}
\end{Highlighting}
\end{Shaded}

\begin{verbatim}
[1] "factor"
\end{verbatim}

\begin{Shaded}
\begin{Highlighting}[]
\NormalTok{>}\StringTok{ }\KeywordTok{levels}\NormalTok{(x)}
\end{Highlighting}
\end{Shaded}

\begin{verbatim}
[1] "x" "y" "z"
\end{verbatim}

\begin{Shaded}
\begin{Highlighting}[]
\NormalTok{>}\StringTok{ }\NormalTok{x[}\DecValTok{2}\NormalTok{] <-}\StringTok{ "c"}                             \CommentTok{# throws a warning}
\end{Highlighting}
\end{Shaded}

\begin{verbatim}
Warning in `[<-.factor`(`*tmp*`, 2, value = "c"): invalid factor level, NA
generated
\end{verbatim}

\begin{Shaded}
\begin{Highlighting}[]
\NormalTok{>}\StringTok{ }\NormalTok{x}
\end{Highlighting}
\end{Shaded}

\begin{verbatim}
[1] x    <NA> z    z    x    z    y   
Levels: x y z
\end{verbatim}

\begin{Shaded}
\begin{Highlighting}[]
\NormalTok{>}\StringTok{ }\NormalTok{x <-}\StringTok{ }\KeywordTok{factor}\NormalTok{(}\KeywordTok{c}\NormalTok{(}\StringTok{"x"}\NormalTok{,}\StringTok{"y"}\NormalTok{,}\StringTok{"z"}\NormalTok{,}\StringTok{"z"}\NormalTok{), }\DataTypeTok{levels =} \KeywordTok{c}\NormalTok{(}\StringTok{"z"}\NormalTok{,}\StringTok{"y"}\NormalTok{,}\StringTok{"x"}\NormalTok{)) }\CommentTok{# specify levels}
\end{Highlighting}
\end{Shaded}

\end{frame}

\begin{frame}[fragile]{Matrices \& arrays}

What separates atomic vectors from arrays is the presence of a
\texttt{dim()} attribute

As arrays are really atomic vectors with this extra attribute, they can
only contain elements of a single type

Matrices are a special case of a multi-dimensional array, where there
are only two dimensions

\begin{Shaded}
\begin{Highlighting}[]
\NormalTok{>}\StringTok{ }\NormalTok{m <-}\StringTok{ }\KeywordTok{matrix}\NormalTok{(}\DecValTok{1}\NormalTok{:}\DecValTok{6}\NormalTok{, }\DataTypeTok{ncol =} \DecValTok{3}\NormalTok{, }\DataTypeTok{nrow =} \DecValTok{2}\NormalTok{)}
\NormalTok{>}\StringTok{ }\NormalTok{m}
\end{Highlighting}
\end{Shaded}

\begin{verbatim}
     [,1] [,2] [,3]
[1,]    1    3    5
[2,]    2    4    6
\end{verbatim}

\begin{Shaded}
\begin{Highlighting}[]
\NormalTok{>}\StringTok{ }\NormalTok{m <-}\StringTok{ }\KeywordTok{matrix}\NormalTok{(}\DecValTok{1}\NormalTok{:}\DecValTok{6}\NormalTok{, }\KeywordTok{c}\NormalTok{(}\DecValTok{3}\NormalTok{,}\DecValTok{2}\NormalTok{))}
\NormalTok{>}\StringTok{ }
\ErrorTok{>}\StringTok{ }\NormalTok{m <-}\StringTok{ }\DecValTok{1}\NormalTok{:}\DecValTok{6}
\NormalTok{>}\StringTok{ }\KeywordTok{dim}\NormalTok{(m) <-}\StringTok{ }\KeywordTok{c}\NormalTok{(}\DecValTok{3}\NormalTok{,}\DecValTok{2}\NormalTok{)}
\NormalTok{>}\StringTok{ }\KeywordTok{dim}\NormalTok{(m) <-}\StringTok{ }\KeywordTok{c}\NormalTok{(}\DecValTok{2}\NormalTok{,}\DecValTok{3}\NormalTok{)}
\NormalTok{>}\StringTok{ }\NormalTok{m}
\end{Highlighting}
\end{Shaded}

\begin{verbatim}
     [,1] [,2] [,3]
[1,]    1    3    5
[2,]    2    4    6
\end{verbatim}

\end{frame}

\begin{frame}[fragile]{Data frames}

Data frames are a bit like Excel worksheets for R; they are like 2-d
matrices with the exception that the columns can store different types
of objects

Internally, data frames are lists, with the extra restriction that each
component (column) of the data frame has to be of the same length

Data frame can be created using \texttt{data.frame()}

\begin{Shaded}
\begin{Highlighting}[]
\NormalTok{>}\StringTok{ }\NormalTok{df <-}\StringTok{ }\KeywordTok{data.frame}\NormalTok{(}\DataTypeTok{x =} \DecValTok{1}\NormalTok{:}\DecValTok{3}\NormalTok{, }\DataTypeTok{y =} \KeywordTok{c}\NormalTok{(}\StringTok{"a"}\NormalTok{, }\StringTok{"b"}\NormalTok{, }\StringTok{"c"}\NormalTok{))}
\NormalTok{>}\StringTok{ }\KeywordTok{str}\NormalTok{(df)}
\end{Highlighting}
\end{Shaded}

\begin{verbatim}
'data.frame':   3 obs. of  2 variables:
 $ x: int  1 2 3
 $ y: Factor w/ 3 levels "a","b","c": 1 2 3
\end{verbatim}

\begin{Shaded}
\begin{Highlighting}[]
\NormalTok{>}\StringTok{ }\NormalTok{df <-}\StringTok{ }\KeywordTok{data.frame}\NormalTok{(}\DataTypeTok{x =} \DecValTok{1}\NormalTok{:}\DecValTok{3}\NormalTok{, }\DataTypeTok{y =} \KeywordTok{c}\NormalTok{(}\StringTok{"a"}\NormalTok{, }\StringTok{"b"}\NormalTok{, }\StringTok{"c"}\NormalTok{), }\DataTypeTok{stringsAsFactors =} \OtherTok{FALSE}\NormalTok{)}
\NormalTok{>}\StringTok{ }\KeywordTok{is.data.frame}\NormalTok{(df)}
\end{Highlighting}
\end{Shaded}

\begin{verbatim}
[1] TRUE
\end{verbatim}

\begin{Shaded}
\begin{Highlighting}[]
\NormalTok{>}\StringTok{ }\KeywordTok{nrow}\NormalTok{(df)}
\end{Highlighting}
\end{Shaded}

\begin{verbatim}
[1] 3
\end{verbatim}

\begin{Shaded}
\begin{Highlighting}[]
\NormalTok{>}\StringTok{ }\KeywordTok{class}\NormalTok{(df)}
\end{Highlighting}
\end{Shaded}

\begin{verbatim}
[1] "data.frame"
\end{verbatim}

\end{frame}

\begin{frame}[fragile]{Data frames II}

Additional columns and rows can be added to a data frame using
\texttt{cbind()} and \texttt{rbind()} respectively

\begin{Shaded}
\begin{Highlighting}[]
\NormalTok{>}\StringTok{ }\KeywordTok{cbind}\NormalTok{(df, }\KeywordTok{data.frame}\NormalTok{(}\DataTypeTok{z =} \DecValTok{3}\NormalTok{:}\DecValTok{1}\NormalTok{))}
\end{Highlighting}
\end{Shaded}

\begin{verbatim}
  x y z
1 1 a 3
2 2 b 2
3 3 c 1
\end{verbatim}

\begin{Shaded}
\begin{Highlighting}[]
\NormalTok{>}\StringTok{ }\KeywordTok{rbind}\NormalTok{(df, }\KeywordTok{data.frame}\NormalTok{(}\DataTypeTok{x =} \DecValTok{10}\NormalTok{, }\DataTypeTok{y =} \StringTok{"z"}\NormalTok{))}
\end{Highlighting}
\end{Shaded}

\begin{verbatim}
   x y
1  1 a
2  2 b
3  3 c
4 10 z
\end{verbatim}

\end{frame}

\begin{frame}{Subsetting}

Subsetting is an incredibly useful and powerful way of manipulating data
objects --- but it is hard to learn initially as there is a lot of
detail to remember

\end{frame}

\begin{frame}[fragile]{Subsetting atomic vectors}

\columnsbegin
\column{0.7\linewidth}

The main subsetting function is \texttt{{[}} --- it takes one of several
types of arguments which determines how the vector is subset

\begin{itemize}
\tightlist
\item
  \emph{positive integers} return the specified elements
\item
  \emph{negative integers} omit the specified elements from the result
\item
  \emph{logical vectors} return elements where the logical vector is
  \texttt{TRUE}
\item
  \emph{nothing} returns the original vector
\item
  \emph{zero} returns a zero-length vector
\end{itemize}

Subsetting a list works the same way with \texttt{{[}}

\column{0.3\linewidth}

\begin{Shaded}
\begin{Highlighting}[]
\NormalTok{>}\StringTok{ }\NormalTok{x <-}\StringTok{ }\KeywordTok{c}\NormalTok{(}\FloatTok{1.3}\NormalTok{, }\FloatTok{4.5}\NormalTok{, }\FloatTok{2.3}\NormalTok{, }\FloatTok{4.2}\NormalTok{, }\FloatTok{5.4}\NormalTok{)}
\NormalTok{>}\StringTok{ }\NormalTok{x[}\KeywordTok{c}\NormalTok{(}\DecValTok{3}\NormalTok{, }\DecValTok{1}\NormalTok{)]}
\end{Highlighting}
\end{Shaded}

\begin{verbatim}
[1] 2.3 1.3
\end{verbatim}

\begin{Shaded}
\begin{Highlighting}[]
\NormalTok{>}\StringTok{ }\NormalTok{x[-}\KeywordTok{c}\NormalTok{(}\DecValTok{3}\NormalTok{, }\DecValTok{1}\NormalTok{)]}
\end{Highlighting}
\end{Shaded}

\begin{verbatim}
[1] 4.5 4.2 5.4
\end{verbatim}

\begin{Shaded}
\begin{Highlighting}[]
\NormalTok{>}\StringTok{ }\NormalTok{x[x >}\StringTok{ }\DecValTok{3}\NormalTok{]}
\end{Highlighting}
\end{Shaded}

\begin{verbatim}
[1] 4.5 4.2 5.4
\end{verbatim}

\begin{Shaded}
\begin{Highlighting}[]
\NormalTok{>}\StringTok{ }\NormalTok{x[}\KeywordTok{c}\NormalTok{(}\OtherTok{TRUE}\NormalTok{, }\OtherTok{FALSE}\NormalTok{)]}
\end{Highlighting}
\end{Shaded}

\begin{verbatim}
[1] 1.3 2.3 5.4
\end{verbatim}

\begin{Shaded}
\begin{Highlighting}[]
\NormalTok{>}\StringTok{ }\NormalTok{x[]}
\end{Highlighting}
\end{Shaded}

\begin{verbatim}
[1] 1.3 4.5 2.3 4.2 5.4
\end{verbatim}

\begin{Shaded}
\begin{Highlighting}[]
\NormalTok{>}\StringTok{ }\NormalTok{x[}\DecValTok{0}\NormalTok{]}
\end{Highlighting}
\end{Shaded}

\begin{verbatim}
numeric(0)
\end{verbatim}

\begin{Shaded}
\begin{Highlighting}[]
\NormalTok{>}\StringTok{ }\NormalTok{(y <-}\StringTok{ }\KeywordTok{setNames}\NormalTok{(x, letters[}\DecValTok{1}\NormalTok{:}\DecValTok{5}\NormalTok{]))}
\end{Highlighting}
\end{Shaded}

\begin{verbatim}
  a   b   c   d   e 
1.3 4.5 2.3 4.2 5.4 
\end{verbatim}

\begin{Shaded}
\begin{Highlighting}[]
\NormalTok{>}\StringTok{ }\NormalTok{y[}\KeywordTok{c}\NormalTok{(}\StringTok{"a"}\NormalTok{, }\StringTok{"c"}\NormalTok{)]}
\end{Highlighting}
\end{Shaded}

\begin{verbatim}
  a   c 
1.3 2.3 
\end{verbatim}

\columnsend

\end{frame}

\begin{frame}[fragile]{Subsetting matrices and arrays}

\columnsbegin
\column{0.7\linewidth}

The main subsetting function is \texttt{{[}} --- it takes one of several
types of arguments which determines how the array is subset

\begin{itemize}
\tightlist
\item
  a pair of 1-d vectors, one for each dimension
\item
  a single 1-d vector
\item
  a matrix
\end{itemize}

\column{0.3\linewidth}

\begin{Shaded}
\begin{Highlighting}[]
\NormalTok{>}\StringTok{ }\NormalTok{a <-}\StringTok{ }\KeywordTok{matrix}\NormalTok{(}\DecValTok{1}\NormalTok{:}\DecValTok{9}\NormalTok{, }\DataTypeTok{nrow =} \DecValTok{3}\NormalTok{)}
\NormalTok{>}\StringTok{ }\KeywordTok{colnames}\NormalTok{(a) <-}\StringTok{ }\NormalTok{LETTERS[}\DecValTok{1}\NormalTok{:}\DecValTok{3}\NormalTok{]}
\NormalTok{>}\StringTok{ }\NormalTok{a[}\DecValTok{1}\NormalTok{:}\DecValTok{2}\NormalTok{, ]}
\end{Highlighting}
\end{Shaded}

\begin{verbatim}
     A B C
[1,] 1 4 7
[2,] 2 5 8
\end{verbatim}

\begin{Shaded}
\begin{Highlighting}[]
\NormalTok{>}\StringTok{ }\NormalTok{a[, }\DecValTok{2}\NormalTok{]}
\end{Highlighting}
\end{Shaded}

\begin{verbatim}
[1] 4 5 6
\end{verbatim}

\columnsend

\end{frame}

\begin{frame}[fragile]{Subsetting data frames}

\columnsbegin
\column{0.5\linewidth}

Again, the main subsetting function is \texttt{{[}} --- you can use it
to subset a data frame as if it were

\begin{itemize}
\tightlist
\item
  a matrix, with an indexing vector for the rows and the columns
\item
  a list
\end{itemize}

\column{0.5\linewidth}

\begin{Shaded}
\begin{Highlighting}[]
\NormalTok{>}\StringTok{ }\NormalTok{df <-}\StringTok{ }\KeywordTok{data.frame}\NormalTok{(}\DataTypeTok{x =} \DecValTok{1}\NormalTok{:}\DecValTok{3}\NormalTok{, }\DataTypeTok{y =} \DecValTok{3}\NormalTok{:}\DecValTok{1}\NormalTok{, }\DataTypeTok{z =} \NormalTok{letters[}\DecValTok{1}\NormalTok{:}\DecValTok{3}\NormalTok{])}
\NormalTok{>}\StringTok{ }\NormalTok{df[df$x ==}\StringTok{ }\DecValTok{2}\NormalTok{, ]}
\end{Highlighting}
\end{Shaded}

\begin{verbatim}
  x y z
2 2 2 b
\end{verbatim}

\begin{Shaded}
\begin{Highlighting}[]
\NormalTok{>}\StringTok{ }\NormalTok{df[}\KeywordTok{c}\NormalTok{(}\StringTok{"x"}\NormalTok{, }\StringTok{"y"}\NormalTok{)]}
\end{Highlighting}
\end{Shaded}

\begin{verbatim}
  x y
1 1 3
2 2 2
3 3 1
\end{verbatim}

\begin{Shaded}
\begin{Highlighting}[]
\NormalTok{>}\StringTok{ }\NormalTok{df[, }\KeywordTok{c}\NormalTok{(}\StringTok{"x"}\NormalTok{, }\StringTok{"y"}\NormalTok{)]}
\end{Highlighting}
\end{Shaded}

\begin{verbatim}
  x y
1 1 3
2 2 2
3 3 1
\end{verbatim}

\begin{Shaded}
\begin{Highlighting}[]
\NormalTok{>}\StringTok{ }\KeywordTok{str}\NormalTok{(df[}\StringTok{"x"}\NormalTok{])}
\end{Highlighting}
\end{Shaded}

\begin{verbatim}
'data.frame':   3 obs. of  1 variable:
 $ x: int  1 2 3
\end{verbatim}

\begin{Shaded}
\begin{Highlighting}[]
\NormalTok{>}\StringTok{ }\KeywordTok{str}\NormalTok{(df[, }\StringTok{"x"}\NormalTok{])                          }\CommentTok{# drops the enpty dimension}
\end{Highlighting}
\end{Shaded}

\begin{verbatim}
 int [1:3] 1 2 3
\end{verbatim}

\columnsend

\end{frame}

\section{Plotting}\label{plotting}

\begin{frame}{Plotting}

R is good at plotting; in-built functionality for producing
publication-ready figures in a range of formats

Three main plotting paradigms in the R ecosystem

\begin{enumerate}
\def\labelenumi{\arabic{enumi}.}
\tightlist
\item
  Base graphics
\item
  \textbf{lattice} package
\item
  \textbf{ggplot2} package
\end{enumerate}

The latter, \textbf{lattice} and \textbf{ggplot}, both use the
underlying \textbf{grid} primitve graphics toolkit

In general, R's graphics are like drawing with pen on paper; once you
draw anything that sheet of paper is no-longer pristine and you can't
erase anything you have drawn

\end{frame}

\begin{frame}[fragile]{Plotting using base graphics}

\columnsbegin
\column{0.5\linewidth}

Standard plotting command is \texttt{plot()}

Takes one or two arguments of coordinates

By default draws a scatterplot

Vast array of parameters to alter look of plots; see \texttt{?par}

Best results often come from building a plot up from bits

\column{0.5\textwidth}

\begin{Shaded}
\begin{Highlighting}[]
\NormalTok{>}\StringTok{ }\NormalTok{x <-}\StringTok{ }\KeywordTok{rnorm}\NormalTok{(}\DecValTok{1000}\NormalTok{)}
\NormalTok{>}\StringTok{ }\NormalTok{y <-}\StringTok{ }\KeywordTok{rnorm}\NormalTok{(}\DecValTok{1000}\NormalTok{)}
\NormalTok{>}\StringTok{ }\KeywordTok{plot}\NormalTok{(x, y)}
\end{Highlighting}
\end{Shaded}

\begin{center}\includegraphics[width=0.95\linewidth]{02-learning-r_files/figure-beamer/base-plot-1-1} \end{center}

\columnsend

\end{frame}

\begin{frame}[fragile]{Plotting using base graphics}

\columnsbegin
\column{0.6\linewidth}

\begin{itemize}
\tightlist
\item
  The \texttt{type} argument changes the type of plotting done

  \begin{itemize}
  \tightlist
  \item
    \texttt{"p"} draws points
  \item
    \texttt{"l"} draws lines
  \item
    \texttt{"o"} draws lines and points over-plotted
  \item
    \texttt{"b"} draws lines and points
  \item
    \texttt{"h"} draws histogram-like bars
  \item
    \texttt{"s"} draws stepped lines
  \end{itemize}
\item
  \texttt{main} control the title of the plot
\item
  \texttt{xlab} \& \texttt{ylab} control axis labels
\end{itemize}

\begin{Shaded}
\begin{Highlighting}[]
\NormalTok{>}\StringTok{ }\NormalTok{x <-}\StringTok{ }\DecValTok{1}\NormalTok{:}\DecValTok{100}
\NormalTok{>}\StringTok{ }\NormalTok{y <-}\StringTok{ }\KeywordTok{cumsum}\NormalTok{(}\KeywordTok{rnorm}\NormalTok{(}\DecValTok{100}\NormalTok{))}
\NormalTok{>}\StringTok{ }\KeywordTok{layout}\NormalTok{(}\KeywordTok{matrix}\NormalTok{(}\DecValTok{1}\NormalTok{:}\DecValTok{2}\NormalTok{, }\DataTypeTok{ncol =} \DecValTok{1}\NormalTok{))}
\NormalTok{>}\StringTok{ }\KeywordTok{plot}\NormalTok{(x, y, }\DataTypeTok{type =} \StringTok{"b"}\NormalTok{, }\DataTypeTok{main =} \StringTok{"a)"}\NormalTok{, }\DataTypeTok{xlab =} \StringTok{"Time"}\NormalTok{, }\DataTypeTok{ylab =} \StringTok{"Random Walk"}\NormalTok{)}
\NormalTok{>}\StringTok{ }\KeywordTok{plot}\NormalTok{(x, y, }\DataTypeTok{type =} \StringTok{"s"}\NormalTok{, }\DataTypeTok{main =} \StringTok{"b)"}\NormalTok{, }\DataTypeTok{xlab =} \StringTok{"Time"}\NormalTok{, }\DataTypeTok{ylab =} \StringTok{"Random Walk"}\NormalTok{)}
\end{Highlighting}
\end{Shaded}

\begin{Shaded}
\begin{Highlighting}[]
\NormalTok{>}\StringTok{ }\KeywordTok{layout}\NormalTok{(}\DecValTok{1}\NormalTok{)}
\end{Highlighting}
\end{Shaded}

\column{0.4\textwidth}

\begin{center}\includegraphics[width=0.6\linewidth]{02-learning-r_files/figure-beamer/base-plot-2a-1} \end{center}

\columnsend

\end{frame}

\begin{frame}[fragile]{Plotting using base graphics}

\columnsbegin
\column{0.6\linewidth}

\begin{itemize}
\tightlist
\item
  \texttt{pch} controls the plotting character
\item
  \texttt{cex} controls the size of the character
\item
  \texttt{col} controls colour
\item
  \texttt{axes} logical; should axes be drawn
\item
  \texttt{ann} logical; should the plot be annotated
\item
  \texttt{axis()}, \texttt{title()}, \texttt{box()} used to build up
  plotting
\item
  Allows finer control
\end{itemize}

\begin{Shaded}
\begin{Highlighting}[]
\NormalTok{>}\StringTok{ }\NormalTok{x <-}\StringTok{ }\KeywordTok{rnorm}\NormalTok{(}\DecValTok{100}\NormalTok{)}
\NormalTok{>}\StringTok{ }\NormalTok{y <-}\StringTok{ }\KeywordTok{rnorm}\NormalTok{(}\DecValTok{100}\NormalTok{)}
\NormalTok{>}\StringTok{ }\KeywordTok{layout}\NormalTok{(}\KeywordTok{matrix}\NormalTok{(}\DecValTok{1}\NormalTok{:}\DecValTok{2}\NormalTok{, }\DataTypeTok{ncol =} \DecValTok{1}\NormalTok{))}
\NormalTok{>}\StringTok{ }\KeywordTok{plot}\NormalTok{(x, y, }\DataTypeTok{main =} \StringTok{"cex = 1.5, pch = 16"}\NormalTok{, }\DataTypeTok{cex =} \FloatTok{1.5}\NormalTok{, }\DataTypeTok{pch =} \DecValTok{19}\NormalTok{, }\DataTypeTok{col =} \StringTok{"red"}\NormalTok{)}
\NormalTok{>}\StringTok{ }\KeywordTok{plot}\NormalTok{(x, y, }\DataTypeTok{cex =} \FloatTok{1.5}\NormalTok{, }\DataTypeTok{pch =} \DecValTok{19}\NormalTok{, }\DataTypeTok{col =} \StringTok{"red"}\NormalTok{, }\DataTypeTok{axes =} \OtherTok{FALSE}\NormalTok{, }\DataTypeTok{ann =} \OtherTok{FALSE}\NormalTok{)}
\NormalTok{>}\StringTok{ }\KeywordTok{axis}\NormalTok{(}\DataTypeTok{side =} \DecValTok{1}\NormalTok{)}
\NormalTok{>}\StringTok{ }\KeywordTok{axis}\NormalTok{(}\DataTypeTok{side =} \DecValTok{2}\NormalTok{)}
\NormalTok{>}\StringTok{ }\KeywordTok{title}\NormalTok{(}\DataTypeTok{main =} \StringTok{"cex = 1.5, pch = 16"}\NormalTok{, }\DataTypeTok{xlab =} \StringTok{"x"}\NormalTok{, }\DataTypeTok{ylab =} \StringTok{"y"}\NormalTok{)}
\NormalTok{>}\StringTok{ }\KeywordTok{box}\NormalTok{()}
\end{Highlighting}
\end{Shaded}

\begin{Shaded}
\begin{Highlighting}[]
\NormalTok{>}\StringTok{ }\KeywordTok{layout}\NormalTok{(}\DecValTok{1}\NormalTok{)}
\end{Highlighting}
\end{Shaded}

\column{0.4\textwidth}

\begin{center}\includegraphics[width=0.6\linewidth]{02-learning-r_files/figure-beamer/base-plot-3a-1} \end{center}

\columnsend

\end{frame}

\begin{frame}[fragile]{Plotting using base graphics}

\columnsbegin
\column{0.6\linewidth}

\begin{itemize}
\tightlist
\item
  \texttt{hist()} draws histograms
\item
  \texttt{boxplot()} draws boxplots
\item
  \texttt{"lwd"} controls the line width
\item
  \texttt{"lty"} controls the line type
\item
  \texttt{lines()} used to add lines to an existing plot
\item
  Also \texttt{points()}
\end{itemize}

\begin{Shaded}
\begin{Highlighting}[]
\NormalTok{>}\StringTok{ }\NormalTok{x <-}\StringTok{ }\KeywordTok{rnorm}\NormalTok{(}\DecValTok{100}\NormalTok{)}
\NormalTok{>}\StringTok{ }\NormalTok{grps <-}\StringTok{ }\KeywordTok{factor}\NormalTok{(}\KeywordTok{sample}\NormalTok{(LETTERS[}\DecValTok{1}\NormalTok{:}\DecValTok{4}\NormalTok{], }\DecValTok{100}\NormalTok{, }\DataTypeTok{replace =} \OtherTok{TRUE}\NormalTok{))}
\NormalTok{>}\StringTok{ }\KeywordTok{layout}\NormalTok{(}\KeywordTok{matrix}\NormalTok{(}\DecValTok{1}\NormalTok{:}\DecValTok{2}\NormalTok{, }\DataTypeTok{ncol =} \DecValTok{1}\NormalTok{))}
\NormalTok{>}\StringTok{ }\NormalTok{dens <-}\KeywordTok{density}\NormalTok{(x)}
\NormalTok{>}\StringTok{ }\KeywordTok{hist}\NormalTok{(x, }\DataTypeTok{freq =} \OtherTok{FALSE}\NormalTok{)}
\NormalTok{>}\StringTok{ }\KeywordTok{lines}\NormalTok{(dens, }\DataTypeTok{col =} \StringTok{"red"}\NormalTok{, }\DataTypeTok{lwd =} \DecValTok{2}\NormalTok{, }\DataTypeTok{lty =} \StringTok{"dashed"}\NormalTok{)}
\NormalTok{>}\StringTok{ }\KeywordTok{boxplot}\NormalTok{(x ~}\StringTok{ }\NormalTok{grps)}
\end{Highlighting}
\end{Shaded}

\begin{Shaded}
\begin{Highlighting}[]
\NormalTok{>}\StringTok{ }\KeywordTok{layout}\NormalTok{(}\DecValTok{1}\NormalTok{)}
\end{Highlighting}
\end{Shaded}

\column{0.4\textwidth}

\begin{center}\includegraphics[width=0.6\linewidth]{02-learning-r_files/figure-beamer/base-plot-4a-1} \end{center}

\columnsend

\end{frame}

\begin{frame}{Plotting device regions and margins}

\begin{center}\includegraphics[width=0.75\linewidth]{02-learning-r_files/figure-beamer/device-regions-1} \end{center}

\end{frame}

\begin{frame}[fragile]{Plotting device regions and margins}

\columnsbegin
\column{0.6\linewidth}

\begin{itemize}
\tightlist
\item
  Control the size of margins using several parameters

  \begin{itemize}
  \tightlist
  \item
    \texttt{mar} --- set margins in terms of number of lines of text
  \item
    \texttt{mai} --- set margins in terms of number of inches
  \end{itemize}
\item
  Specify as a vector of length 4 ---
  \texttt{mar\ =\ c(5,4,4,2)\ +\ 0.1)}
\item
  The ordering is Bottom, Left, Top, Right
\item
  The outer margin is controlled via parameter \texttt{oma} and
  \texttt{omi}, just like \texttt{mar}
\item
  By default, there is no outer margin --- \texttt{oma\ =\ c(0,0,0,0)}
\end{itemize}

\column{0.4\linewidth}

\begin{Shaded}
\begin{Highlighting}[]
\NormalTok{>}\StringTok{ }\NormalTok{x <-}\StringTok{ }\KeywordTok{runif}\NormalTok{(}\DecValTok{100}\NormalTok{)}
\NormalTok{>}\StringTok{ }\NormalTok{y <-}\StringTok{ }\DecValTok{4} \NormalTok{+}\StringTok{ }\NormalTok{(}\FloatTok{2.1} \NormalTok{*}\StringTok{ }\NormalTok{x) +}\StringTok{ }\KeywordTok{rnorm}\NormalTok{(}\DecValTok{100}\NormalTok{, }\DecValTok{0}\NormalTok{, }\DecValTok{3}\NormalTok{)}
\NormalTok{>}\StringTok{ }\NormalTok{op <-}\StringTok{ }\KeywordTok{par}\NormalTok{(}\DataTypeTok{mar =} \KeywordTok{c}\NormalTok{(}\DecValTok{4}\NormalTok{,}\DecValTok{4}\NormalTok{,}\DecValTok{4}\NormalTok{,}\DecValTok{4}\NormalTok{) +}\StringTok{ }\FloatTok{0.1}\NormalTok{)}
\NormalTok{>}\StringTok{ }\KeywordTok{plot}\NormalTok{(y ~}\StringTok{ }\NormalTok{x)}
\end{Highlighting}
\end{Shaded}

\begin{Shaded}
\begin{Highlighting}[]
\NormalTok{>}\StringTok{ }\NormalTok{op <-}\StringTok{ }\KeywordTok{par}\NormalTok{(op)}
\NormalTok{>}\StringTok{ }
\ErrorTok{>}\StringTok{ }\NormalTok{x <-}\StringTok{ }\KeywordTok{runif}\NormalTok{(}\DecValTok{100}\NormalTok{)}
\NormalTok{>}\StringTok{ }\NormalTok{y <-}\StringTok{ }\DecValTok{4} \NormalTok{+}\StringTok{ }\NormalTok{(}\FloatTok{2.1} \NormalTok{*}\StringTok{ }\NormalTok{x) +}\StringTok{ }\KeywordTok{rnorm}\NormalTok{(}\DecValTok{100}\NormalTok{, }\DecValTok{0}\NormalTok{, }\DecValTok{3}\NormalTok{)}
\NormalTok{>}\StringTok{ }\NormalTok{op <-}\StringTok{ }\KeywordTok{par}\NormalTok{(}\DataTypeTok{mar =} \KeywordTok{c}\NormalTok{(}\DecValTok{4}\NormalTok{,}\DecValTok{4}\NormalTok{,}\DecValTok{4}\NormalTok{,}\DecValTok{4}\NormalTok{) +}\StringTok{ }\FloatTok{0.1}\NormalTok{, }\DataTypeTok{oma =} \KeywordTok{rep}\NormalTok{(}\DecValTok{2}\NormalTok{,}\DecValTok{4}\NormalTok{))}
\NormalTok{>}\StringTok{ }\KeywordTok{plot}\NormalTok{(y ~}\StringTok{ }\NormalTok{x)}
\end{Highlighting}
\end{Shaded}

\begin{Shaded}
\begin{Highlighting}[]
\NormalTok{>}\StringTok{ }\NormalTok{op <-}\StringTok{ }\KeywordTok{par}\NormalTok{(op)}
\end{Highlighting}
\end{Shaded}

\columnsend

\end{frame}

\begin{frame}[fragile]{Setting graphical parameters}

\begin{itemize}
\tightlist
\item
  Base graphic are controlled by a large number of plotting graphical
  parameters
\item
  These are detailed in the help page \texttt{?par}
\item
  Graphical parameters are changed using the \texttt{par()} function and
  some may be changed within plotting calls
\item
  To avoid getting into a muddle, when changing par you should

  \begin{itemize}
  \tightlist
  \item
    Store the defaults
  \item
    Change your parameters as required
  \item
    When finished the current plot, reset the parameters
  \end{itemize}
\item
  The first two can be done with a single R call
\end{itemize}

\begin{Shaded}
\begin{Highlighting}[]
\NormalTok{>}\StringTok{ }\NormalTok{## Store defaults in 'op' and change current parameters}
\ErrorTok{>}\StringTok{ }\NormalTok{op <-}\StringTok{ }\KeywordTok{par}\NormalTok{(}\DataTypeTok{las =} \DecValTok{2}\NormalTok{, }\DataTypeTok{mar =} \KeywordTok{rep}\NormalTok{(}\DecValTok{4}\NormalTok{, }\DecValTok{4}\NormalTok{), }\DataTypeTok{oma =} \KeywordTok{c}\NormalTok{(}\DecValTok{1}\NormalTok{,}\DecValTok{3}\NormalTok{,}\DecValTok{4}\NormalTok{,}\DecValTok{2}\NormalTok{), }\DataTypeTok{cex.main =} \DecValTok{2}\NormalTok{)}
\NormalTok{>}\StringTok{ }\KeywordTok{plot}\NormalTok{(}\DecValTok{1}\NormalTok{:}\DecValTok{10}\NormalTok{) }\CommentTok{# plot something}
\end{Highlighting}
\end{Shaded}

\begin{Shaded}
\begin{Highlighting}[]
\NormalTok{>}\StringTok{ }\KeywordTok{par}\NormalTok{(op)    }\CommentTok{# reset}
\end{Highlighting}
\end{Shaded}

\end{frame}

\begin{frame}{Plotting on multiple device regions}

\begin{center}\includegraphics[width=0.4\linewidth]{02-learning-r_files/figure-beamer/multiple-regions-1} \end{center}

\end{frame}

\begin{frame}[fragile]{Plotting on multiple device regions}

\columnsbegin
\column{0.6\linewidth}

Several ways to split a region into multiple plotting regions

\begin{itemize}
\tightlist
\item
  Graphical parameters \texttt{mfrow} \& \texttt{mfcol}
\item
  The \texttt{layout()} function
\item
  The \texttt{split.screen()} function
\end{itemize}

\column{0.4\linewidth}

\begin{Shaded}
\begin{Highlighting}[]
\NormalTok{>}\StringTok{ }\NormalTok{op <-}\StringTok{ }\KeywordTok{par}\NormalTok{(}\DataTypeTok{mfrow =} \KeywordTok{c}\NormalTok{(}\DecValTok{2}\NormalTok{,}\DecValTok{2}\NormalTok{))}
\NormalTok{>}\StringTok{ }\KeywordTok{plot}\NormalTok{(}\DecValTok{1}\NormalTok{:}\DecValTok{10}\NormalTok{)}
\NormalTok{>}\StringTok{ }\KeywordTok{plot}\NormalTok{(}\DecValTok{1}\NormalTok{:}\DecValTok{10}\NormalTok{)}
\NormalTok{>}\StringTok{ }\KeywordTok{plot}\NormalTok{(}\DecValTok{1}\NormalTok{:}\DecValTok{10}\NormalTok{)}
\NormalTok{>}\StringTok{ }\KeywordTok{plot}\NormalTok{(}\DecValTok{1}\NormalTok{:}\DecValTok{10}\NormalTok{)}
\NormalTok{>}\StringTok{ }\KeywordTok{par}\NormalTok{(op)}
\end{Highlighting}
\end{Shaded}

\begin{center}\includegraphics[width=\linewidth]{02-learning-r_files/figure-beamer/mfrow-1} \end{center}

\columnsend

\end{frame}

\begin{frame}[fragile]{Plotting on multiple device regions}

\columnsbegin
\column{0.6\linewidth}

\texttt{layout()} allows you to

\begin{itemize}
\tightlist
\item
  break the device into multiple reqions with individual regions not
  tied to a single row or column
\item
  specify where each plot should go via an ID
\end{itemize}

Previous figure could have been produced using
\texttt{layout(matrix(1:4,\ ncol\ =\ 2,\ byrow\ =\ TRUE))}

Here, we allow the whole first row to be occupied by region 1

\column{0.4\linewidth}

\begin{Shaded}
\begin{Highlighting}[]
\NormalTok{>}\StringTok{ }\KeywordTok{layout}\NormalTok{(}\KeywordTok{matrix}\NormalTok{(}\KeywordTok{c}\NormalTok{(}\DecValTok{1}\NormalTok{,}\DecValTok{1}\NormalTok{,}\DecValTok{2}\NormalTok{,}\DecValTok{3}\NormalTok{), }\DataTypeTok{ncol =} \DecValTok{2}\NormalTok{, }\DataTypeTok{byrow =} \OtherTok{TRUE}\NormalTok{))}
\NormalTok{>}\StringTok{ }\KeywordTok{plot}\NormalTok{(}\DecValTok{1}\NormalTok{:}\DecValTok{10}\NormalTok{)}
\NormalTok{>}\StringTok{ }\KeywordTok{plot}\NormalTok{(}\DecValTok{1}\NormalTok{:}\DecValTok{10}\NormalTok{)}
\NormalTok{>}\StringTok{ }\KeywordTok{plot}\NormalTok{(}\DecValTok{1}\NormalTok{:}\DecValTok{10}\NormalTok{)}
\NormalTok{>}\StringTok{ }\KeywordTok{layout}\NormalTok{(}\DecValTok{1}\NormalTok{)}
\end{Highlighting}
\end{Shaded}

\begin{center}\includegraphics[width=\linewidth]{02-learning-r_files/figure-beamer/layout-1} \end{center}

\columnsend

\end{frame}

\section{ggplot}\label{ggplot}

\begin{frame}[fragile]{ggplot}

\columnsbegin
\column{0.6\linewidth}

\alert{ggplot2} is what all the cool, young kids are using

High-level plotting package like Lattice, but designed for ease of use

Based on the Leland Wilkinson's \emph{Grammar of Graphics}

Three key principles

\begin{itemize}
\tightlist
\item
  a tidy data structure
\item
  map data to plot features --- \alert{aesthetics}
\item
  display data using geometric objects --- \alert{geoms}
\end{itemize}

Plots are built up in layers, composed with \texttt{+}

\column{0.4\linewidth}

\begin{Shaded}
\begin{Highlighting}[]
\NormalTok{>}\StringTok{ }\NormalTok{df <-}\StringTok{ }\KeywordTok{data.frame}\NormalTok{(}\DataTypeTok{x =} \DecValTok{1}\NormalTok{:}\DecValTok{100}\NormalTok{, }\DataTypeTok{y =} \KeywordTok{cumsum}\NormalTok{(}\KeywordTok{rnorm}\NormalTok{(}\DecValTok{100}\NormalTok{)))}
\NormalTok{>}\StringTok{ }\KeywordTok{ggplot}\NormalTok{(df, }\KeywordTok{aes}\NormalTok{(}\DataTypeTok{x =} \NormalTok{x, }\DataTypeTok{y =} \NormalTok{y)) +}
\NormalTok{+}\StringTok{     }\KeywordTok{geom_point}\NormalTok{() +}
\NormalTok{+}\StringTok{     }\KeywordTok{geom_smooth}\NormalTok{(}\DataTypeTok{span =} \FloatTok{0.3}\NormalTok{)}
\end{Highlighting}
\end{Shaded}

\begin{center}\includegraphics[width=\linewidth]{02-learning-r_files/figure-beamer/ggplot-1-1} \end{center}

\columnsend

\end{frame}

\begin{frame}[fragile]{ggplot --- mapping \& aesthetics}

\columnsbegin
\column{0.6\linewidth}

Variables in the data object are \emph{mapped} to aesthetics that you
perceive on the plot

Commonly used aesthetics are

\begin{itemize}
\tightlist
\item
  \texttt{x} and \texttt{y}
\item
  \texttt{color} (or \texttt{colour})
\item
  \texttt{shape}
\item
  \texttt{size}
\item
  \texttt{fill} (for anything covering an area, not points)
\end{itemize}

Specified via the \texttt{aes()} function

\textbf{ggplot} handles the creation of legends (keys) for you

\column{0.4\linewidth}

\begin{Shaded}
\begin{Highlighting}[]
\NormalTok{>}\StringTok{ }\KeywordTok{data}\NormalTok{(diamonds)}
\NormalTok{>}\StringTok{ }\KeywordTok{set.seed}\NormalTok{(}\DecValTok{1410}\NormalTok{)}
\NormalTok{>}\StringTok{ }\NormalTok{dsmall <-}\StringTok{ }\NormalTok{diamonds[}\KeywordTok{sample}\NormalTok{(}\KeywordTok{nrow}\NormalTok{(diamonds), }\DecValTok{100}\NormalTok{), ]}
\NormalTok{>}\StringTok{ }\KeywordTok{ggplot}\NormalTok{(dsmall, }\KeywordTok{aes}\NormalTok{(}\DataTypeTok{x =} \NormalTok{carat, }\DataTypeTok{y =} \NormalTok{price, }\DataTypeTok{colour =} \NormalTok{color, }\DataTypeTok{shape =} \NormalTok{cut)) +}
\NormalTok{+}\StringTok{     }\KeywordTok{geom_point}\NormalTok{()}
\end{Highlighting}
\end{Shaded}

\begin{center}\includegraphics[width=\linewidth]{02-learning-r_files/figure-beamer/ggplot-2-1} \end{center}

\columnsend

\end{frame}

\begin{frame}[fragile]{ggplot --- geoms; geometric objects}

Geometric objects control the way the data are represented on the plot

Common geoms include:

\begin{itemize}
\tightlist
\item
  \texttt{geom\ =\ "point"} --- scatterplot
\item
  \texttt{geom\ =\ "smooth"} --- fits a smooth to the data and draws the
  smooth and its standard error
\item
  \texttt{geom\ =\ "boxplot"} --- box plots
\item
  \texttt{geom\ =\ "line"} and \texttt{geom\ =\ "path"} produce line
  plots. \texttt{"line"} produces lines from left to right, whilst
  \texttt{"path"} draws in data order
\item
  \texttt{geom\ =\ "histogram"} --- histograms
\item
  \texttt{geom\ =\ "freqpoly"} --- frequency polygons
\item
  \texttt{geom\ =\ "density"} --- density plots
\item
  \texttt{geom\ =\ "bar"} --- bar plots
\end{itemize}

\end{frame}

\begin{frame}[fragile]{ggplot}

\columnsbegin
\column{0.6\linewidth}

If we have grouping variable(s), we can map them to aesthetics or we can
\alert{facet} the plot to produce multiple panels

Already seen how mapping works

Here we create a random time series for each of four hypothetical sites

\column{0.4\linewidth}

\begin{Shaded}
\begin{Highlighting}[]
\NormalTok{>}\StringTok{ }\KeywordTok{set.seed}\NormalTok{(}\DecValTok{789}\NormalTok{)}
\NormalTok{>}\StringTok{ }\NormalTok{dat2 <-}\StringTok{ }\KeywordTok{data.frame}\NormalTok{(}\DataTypeTok{x =} \KeywordTok{rep}\NormalTok{(}\DecValTok{1}\NormalTok{:}\DecValTok{1000}\NormalTok{, }\DecValTok{4}\NormalTok{),}
\NormalTok{+}\StringTok{                    }\DataTypeTok{y =} \KeywordTok{cumsum}\NormalTok{(}\KeywordTok{rnorm}\NormalTok{(}\DecValTok{1000}\NormalTok{*}\DecValTok{4}\NormalTok{)),}
\NormalTok{+}\StringTok{                    }\DataTypeTok{Site =} \KeywordTok{factor}\NormalTok{(}\KeywordTok{rep}\NormalTok{(LETTERS[}\DecValTok{1}\NormalTok{:}\DecValTok{4}\NormalTok{], }\DataTypeTok{each =} \DecValTok{1000}\NormalTok{)))}
\NormalTok{>}\StringTok{ }\KeywordTok{ggplot}\NormalTok{(dat2, }\KeywordTok{aes}\NormalTok{(}\DataTypeTok{x =} \NormalTok{x, }\DataTypeTok{y =} \NormalTok{y, }\DataTypeTok{colour =} \NormalTok{Site, }\DataTypeTok{fill =} \NormalTok{Site)) +}
\NormalTok{+}\StringTok{     }\KeywordTok{geom_line}\NormalTok{() +}\StringTok{ }\KeywordTok{geom_smooth}\NormalTok{() +}
\NormalTok{+}\StringTok{     }\KeywordTok{labs}\NormalTok{(}\DataTypeTok{x =} \StringTok{"Time"}\NormalTok{, }\DataTypeTok{y =} \StringTok{"Value"}\NormalTok{)}
\end{Highlighting}
\end{Shaded}

\begin{center}\includegraphics[width=\linewidth]{02-learning-r_files/figure-beamer/ggplot-multiple-time-series-1} \end{center}

\columnsend

\end{frame}

\begin{frame}[fragile]{ggplot --- facetting}

\columnsbegin
\column{0.6\linewidth}

\alert{Facetting} is what \textbf{ggplot} calls the process of breaking
up data via one or more grouping variable and displaying a panel for
each group that shows the data for that group

Two types of facetting

\begin{itemize}
\tightlist
\item
  \texttt{facet\_wrap()} --- wraps facets into a tabular arrangement
\item
  \texttt{facet\_grid()} --- arranges facets by 1 or 2 categorical
  variables assigned to the rows and columns of the grid
\end{itemize}

\texttt{facet\_wrap()} takes a one-sided formula

\texttt{facet\_grid()} takes a two-sided formula

\column{0.4\linewidth}

\begin{Shaded}
\begin{Highlighting}[]
\NormalTok{>}\StringTok{ }\KeywordTok{ggplot}\NormalTok{(dat2, }\KeywordTok{aes}\NormalTok{(}\DataTypeTok{x =} \NormalTok{x, }\DataTypeTok{y =} \NormalTok{y, }\DataTypeTok{colour =} \NormalTok{Site, }\DataTypeTok{fill =} \NormalTok{Site)) +}
\NormalTok{+}\StringTok{     }\KeywordTok{geom_line}\NormalTok{() +}\StringTok{ }\KeywordTok{geom_smooth}\NormalTok{() +}
\NormalTok{+}\StringTok{     }\KeywordTok{labs}\NormalTok{(}\DataTypeTok{x =} \StringTok{"Time"}\NormalTok{, }\DataTypeTok{y =} \StringTok{"Value"}\NormalTok{) +}
\NormalTok{+}\StringTok{     }\KeywordTok{facet_wrap}\NormalTok{( ~}\StringTok{ }\NormalTok{Site, }\DataTypeTok{ncol =} \DecValTok{2}\NormalTok{)}
\end{Highlighting}
\end{Shaded}

\begin{center}\includegraphics[width=\linewidth]{02-learning-r_files/figure-beamer/ggplot-facet-wrap-1} \end{center}

\columnsend

\end{frame}

\begin{frame}[fragile]{ggplot --- facetting}

\columnsbegin
\column{0.6\linewidth}

\alert{Facetting} is what \textbf{ggplot} calls the process of breaking
up data via one or more grouping variable and displaying a panel for
each group that shows the data for that group

Two types of facetting

\begin{itemize}
\tightlist
\item
  \texttt{facet\_wrap()} --- wraps facets into a tabular arrangement
\item
  \texttt{facet\_grid()} --- arranges facets by 1 or 2 categorical
  variables assigned to the rows and columns of the grid
\end{itemize}

\texttt{facet\_wrap()} takes a one-sided formula

\texttt{facet\_grid()} takes a two-sided formula

\column{0.4\linewidth}

\begin{Shaded}
\begin{Highlighting}[]
\NormalTok{>}\StringTok{ }\KeywordTok{ggplot}\NormalTok{(dat2, }\KeywordTok{aes}\NormalTok{(}\DataTypeTok{x =} \NormalTok{x, }\DataTypeTok{y =} \NormalTok{y, }\DataTypeTok{colour =} \NormalTok{Site, }\DataTypeTok{fill =} \NormalTok{Site)) +}
\NormalTok{+}\StringTok{     }\KeywordTok{geom_line}\NormalTok{() +}\StringTok{ }\KeywordTok{geom_smooth}\NormalTok{() +}
\NormalTok{+}\StringTok{     }\KeywordTok{labs}\NormalTok{(}\DataTypeTok{x =} \StringTok{"Time"}\NormalTok{, }\DataTypeTok{y =} \StringTok{"Value"}\NormalTok{) +}
\NormalTok{+}\StringTok{     }\KeywordTok{facet_grid}\NormalTok{(Site ~}\StringTok{ }\NormalTok{.) +}
\NormalTok{+}\StringTok{     }\KeywordTok{theme}\NormalTok{(}\DataTypeTok{legend.position =} \StringTok{"top"}\NormalTok{)}
\end{Highlighting}
\end{Shaded}

\begin{center}\includegraphics[width=\linewidth]{02-learning-r_files/figure-beamer/ggplot-facet-grid-1} \end{center}

\columnsend

\end{frame}

\section{R package management}\label{r-package-management}

\begin{frame}[fragile]{R package management}

\begin{itemize}
\tightlist
\item
  CRAN contains hundreds of packages of user-contributed code that you
  can install from an R session
\item
  Package installation via function \texttt{install.packages()}
\item
  Packages can be updated via function \texttt{updates.packages()}
\item
  When installing or updating for the first time in a session, R will
  prompt you to choose a mirror to download from
\item
  Once a package is installed you need to load it ready for use
\item
  Load a package from your library using \texttt{library()} or
  \texttt{require()}
\item
  Windows and MacOS have menu items to assist with these operations as
  does RStudio
\end{itemize}

\begin{Shaded}
\begin{Highlighting}[]
\NormalTok{>}\StringTok{ }\KeywordTok{install.packages}\NormalTok{(}\StringTok{"vegan"}\NormalTok{)}
\NormalTok{>}\StringTok{ }\KeywordTok{update.packages}\NormalTok{()}
\NormalTok{>}\StringTok{ }\KeywordTok{update.packages}\NormalTok{(}\DataTypeTok{ask =} \OtherTok{FALSE}\NormalTok{)}
\NormalTok{>}\StringTok{ }\KeywordTok{library}\NormalTok{(}\StringTok{"vegan"}\NormalTok{)}
\end{Highlighting}
\end{Shaded}

\end{frame}

\begin{frame}[fragile]{R package management}

\begin{itemize}
\tightlist
\item
  It is useful to create your own library for downloaded packages
\item
  This library will not be overwritten when you install a new version of
  R
\item
  To set a directory you have write permissions on as your user library,
  create a file named \texttt{.Renviron} in your home directory

  \begin{itemize}
  \tightlist
  \item
    On Windows this is usually
    \texttt{C:\textbackslash{}Documents\textasciitilde{}and\textasciitilde{}Settings\textbackslash{}username\textbackslash{}My\textasciitilde{}Documents}
  \item
    On Linux it is \texttt{/home/user/}
  \end{itemize}
\item
  To set your user library to stated directory, add following to your
  \texttt{.Renviron}

  \begin{itemize}
  \tightlist
  \item
    On Windows if installed R to \texttt{C:\textbackslash{}R} add:
    \texttt{R\_LIBS=C:/R/myRlib}
  \item
    On Linux, create directory \texttt{/home/user/R/libs} say and then
    add: \texttt{R\_LIBS=/home/user/R/libs}
  \end{itemize}
\end{itemize}

\end{frame}

\begin{frame}{Re-use}

Copyright © (2015--2017) Gavin L. Simpson Some Rights Reserved

Unless indicated otherwise, this slide deck is licensed under a
\href{http://creativecommons.org/licenses/by/4.0/}{Creative Commons
Attribution 4.0 International License}.

\begin{center}
  \ccby
\end{center}

\end{frame}

\end{document}
