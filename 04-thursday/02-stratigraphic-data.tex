\documentclass[10pt,ignorenonframetext,compress, aspectratio=169]{beamer}
\setbeamertemplate{caption}[numbered]
\setbeamertemplate{caption label separator}{: }
\setbeamercolor{caption name}{fg=normal text.fg}
\beamertemplatenavigationsymbolsempty
\usepackage{lmodern}
\usepackage{amssymb,amsmath,mathtools}
\usepackage{ifxetex,ifluatex}
\usepackage{fixltx2e} % provides \textsubscript
\ifnum 0\ifxetex 1\fi\ifluatex 1\fi=0 % if pdftex
  \usepackage[T1]{fontenc}
  \usepackage[utf8]{inputenc}
\else % if luatex or xelatex
  \ifxetex
    \usepackage{mathspec}
  \else
    \usepackage{fontspec}
  \fi
  %%\defaultfontfeatures{Ligatures=TeX,Scale=MatchLowercase}
  \defaultfontfeatures{Scale=MatchLowercase}
\fi
\usetheme[]{metropolis}
% use upquote if available, for straight quotes in verbatim environments
\IfFileExists{upquote.sty}{\usepackage{upquote}}{}
% use microtype if available
\IfFileExists{microtype.sty}{%
\usepackage{microtype}
\UseMicrotypeSet[protrusion]{basicmath} % disable protrusion for tt fonts
}{}
\newif\ifbibliography
\usepackage{color}
\usepackage{fancyvrb}
\newcommand{\VerbBar}{|}
\newcommand{\VERB}{\Verb[commandchars=\\\{\}]}
\DefineVerbatimEnvironment{Highlighting}{Verbatim}{commandchars=\\\{\}}
% Add ',fontsize=\small' for more characters per line
\usepackage{framed}
\definecolor{shadecolor}{RGB}{248,248,248}
\newenvironment{Shaded}{\begin{snugshade}}{\end{snugshade}}
\newcommand{\KeywordTok}[1]{\textcolor[rgb]{0.13,0.29,0.53}{\textbf{{#1}}}}
\newcommand{\DataTypeTok}[1]{\textcolor[rgb]{0.13,0.29,0.53}{{#1}}}
\newcommand{\DecValTok}[1]{\textcolor[rgb]{0.00,0.00,0.81}{{#1}}}
\newcommand{\BaseNTok}[1]{\textcolor[rgb]{0.00,0.00,0.81}{{#1}}}
\newcommand{\FloatTok}[1]{\textcolor[rgb]{0.00,0.00,0.81}{{#1}}}
\newcommand{\ConstantTok}[1]{\textcolor[rgb]{0.00,0.00,0.00}{{#1}}}
\newcommand{\CharTok}[1]{\textcolor[rgb]{0.31,0.60,0.02}{{#1}}}
\newcommand{\SpecialCharTok}[1]{\textcolor[rgb]{0.00,0.00,0.00}{{#1}}}
\newcommand{\StringTok}[1]{\textcolor[rgb]{0.31,0.60,0.02}{{#1}}}
\newcommand{\VerbatimStringTok}[1]{\textcolor[rgb]{0.31,0.60,0.02}{{#1}}}
\newcommand{\SpecialStringTok}[1]{\textcolor[rgb]{0.31,0.60,0.02}{{#1}}}
\newcommand{\ImportTok}[1]{{#1}}
\newcommand{\CommentTok}[1]{\textcolor[rgb]{0.56,0.35,0.01}{\textit{{#1}}}}
\newcommand{\DocumentationTok}[1]{\textcolor[rgb]{0.56,0.35,0.01}{\textbf{\textit{{#1}}}}}
\newcommand{\AnnotationTok}[1]{\textcolor[rgb]{0.56,0.35,0.01}{\textbf{\textit{{#1}}}}}
\newcommand{\CommentVarTok}[1]{\textcolor[rgb]{0.56,0.35,0.01}{\textbf{\textit{{#1}}}}}
\newcommand{\OtherTok}[1]{\textcolor[rgb]{0.56,0.35,0.01}{{#1}}}
\newcommand{\FunctionTok}[1]{\textcolor[rgb]{0.00,0.00,0.00}{{#1}}}
\newcommand{\VariableTok}[1]{\textcolor[rgb]{0.00,0.00,0.00}{{#1}}}
\newcommand{\ControlFlowTok}[1]{\textcolor[rgb]{0.13,0.29,0.53}{\textbf{{#1}}}}
\newcommand{\OperatorTok}[1]{\textcolor[rgb]{0.81,0.36,0.00}{\textbf{{#1}}}}
\newcommand{\BuiltInTok}[1]{{#1}}
\newcommand{\ExtensionTok}[1]{{#1}}
\newcommand{\PreprocessorTok}[1]{\textcolor[rgb]{0.56,0.35,0.01}{\textit{{#1}}}}
\newcommand{\AttributeTok}[1]{\textcolor[rgb]{0.77,0.63,0.00}{{#1}}}
\newcommand{\RegionMarkerTok}[1]{{#1}}
\newcommand{\InformationTok}[1]{\textcolor[rgb]{0.56,0.35,0.01}{\textbf{\textit{{#1}}}}}
\newcommand{\WarningTok}[1]{\textcolor[rgb]{0.56,0.35,0.01}{\textbf{\textit{{#1}}}}}
\newcommand{\AlertTok}[1]{\textcolor[rgb]{0.94,0.16,0.16}{{#1}}}
\newcommand{\ErrorTok}[1]{\textcolor[rgb]{0.64,0.00,0.00}{\textbf{{#1}}}}
\newcommand{\NormalTok}[1]{{#1}}

% Prevent slide breaks in the middle of a paragraph:
\widowpenalties 1 10000
\raggedbottom

\AtBeginPart{
  \let\insertpartnumber\relax
  \let\partname\relax
  \frame{\partpage}
}
\AtBeginSection{
  \ifbibliography
  \else
    \let\insertsectionnumber\relax
    \let\sectionname\relax
    \frame{\sectionpage}
  \fi
}
\AtBeginSubsection{
  \let\insertsubsectionnumber\relax
  \let\subsectionname\relax
  \frame{\subsectionpage}
}

\setlength{\parindent}{0pt}
\setlength{\parskip}{6pt plus 2pt minus 1pt}
\setlength{\emergencystretch}{3em}  % prevent overfull lines
\providecommand{\tightlist}{%
  \setlength{\itemsep}{0pt}\setlength{\parskip}{0pt}}
\setcounter{secnumdepth}{0}

%% GLS Added
% Textcomp for various common symbols
\usepackage{textcomp}

\usepackage{booktabs}

% Creative Commons Icons
\usepackage[scale=1]{ccicons}

\newenvironment{centrefig}{\begin{figure}\centering}{\end{figure}}
\newcommand{\columnsbegin}{\begin{columns}}
\newcommand{\columnsend}{\end{columns}}
\newcommand{\centreFigBegin}{\begin{figure}\centering}
\newcommand{\centreFigEnd}{\end{figure}}
%%

\DefineVerbatimEnvironment{Highlighting}{Verbatim}{commandchars=\\\{\}, fontsize=\tiny}
% make console-output smaller:
\makeatletter
\def\verbatim{\tiny\@verbatim \frenchspacing\@vobeyspaces \@xverbatim}
\makeatother
%\setlength{\parskip}{0pt}
\setlength{\OuterFrameSep}{-4pt} % was -4pt
\makeatletter
\preto{\@verbatim}{\topsep=-10pt \partopsep=-10pt} % were 10pt
\makeatother

\title{Stratigraphic Data}
\author{Gavin L. Simpson}
\date{U Adelaide 2017 • Feb 13--17 2017}

\begin{document}
\frame{\titlepage}

\section{Introduction}\label{introduction}

\begin{frame}{Introduction}

\end{frame}

\section{Summarising stratigraphic
data}\label{summarising-stratigraphic-data}

\begin{frame}{Summarising change in stratigraphic data}

\begin{itemize}
\tightlist
\item
  Ordination commonly used to describe patterns of change in
  multivariate sediment core data
\item
  PCA, CA, or even DCA axis 1 and 2 scores commonly used
\item
  These methods capture largest patterns of variation in underlying data
  under assumption of particular model
\item
  Can be upset by data sets with a dominant gradient
\item
  Can apply all techniques learned earlier in workshop to stratigraphic
  data
\item
  Can we do any better than these methods?
\end{itemize}

\end{frame}

\begin{frame}{Principal Curves}

A single long or dominant gradient in an (palaeo)ecological data set
poses particular problems for PCA and CA --- \alert{horseshoe} or
\alert{arch}

\begin{center}\includegraphics[width=0.7\linewidth]{02-stratigraphic-data_files/figure-beamer/plot-abernethy-1} \end{center}

Abernethy Forest pollen data (Birks \& Mathewes, 1978)

\end{frame}

\begin{frame}{Principal Curves}

A single long or dominant gradient in an (palaeo)ecological data set
poses particular problems for PCA and CA --- \alert{horseshoe} or
\alert{arch}

Trend is broken over two or more axes:

\begin{center}\includegraphics[width=0.7\linewidth]{02-stratigraphic-data_files/figure-beamer/plot-abernethy-pca-1} \end{center}

Can we generalise the PCA solution to be a smooth, non-linear surface?

\end{frame}

\begin{frame}{Principal Curves --- Comparison of estimation techniques}

\begin{itemize}
\tightlist
\item
  In OLS regression, \(y\) is the response and \(x\) is assumed without
  error
\item
  Errors are minimised in \(y\) only --- sums of squared errors
\item
  PCA can be seen as a regression of \(y\) on \(x\) where neither \(y\)
  nor \(x\) plays the role of response or predictor
\item
  In PCA, errors in both \(x\) and \(y\) are minimised --- sums of
  squared orthogonal errors
\end{itemize}

\end{frame}

\begin{frame}{Principal Curves --- Comparison of estimation techniques}

\begin{center}\includegraphics[width=0.55\linewidth]{02-stratigraphic-data_files/figure-beamer/comparison-figure-1} \end{center}

\end{frame}

\begin{frame}{Principal Curves --- Comparison of estimation techniques}

We can generalise the OLS model to a regression of \(y\) on \(x\) using
a smooth function of \(x\), \(f(x)\), as the predictor

\(f(x)\) can be estimated using a multitude of techniques

\begin{itemize}
\tightlist
\item
  Loess smoother
\item
  Local-linear smooths
\item
  Smoothing splines
\item
  Regression splines
\item
  \ldots{}
\end{itemize}

\(f(x)\) is usually estimated from the data, with smoothness determined
by minimising a penalised sums of squares criterion under CV (or GCV):
\textbf{Errors are still assessed in \(y\) only}

\end{frame}

\begin{frame}{Principal Curves --- Comparison of estimation techniques}

\begin{center}\includegraphics[width=0.55\linewidth]{02-stratigraphic-data_files/figure-beamer/comparison-figure-2-1} \end{center}

\end{frame}

\begin{frame}{Principal Curves --- Comparison of estimation techniques}

\begin{itemize}
\tightlist
\item
  Ideally we would generalise PCA to find non-linear manifolds in the
  same way as we went from OLS to semi-parametric regression using
  smoothers
\item
  This is exactly what is done in the method of \alert{principal curves}
\item
  Our aim is to estimate as the principal curve, a 1-d manifold that
  passes through the data in high-dimensions that minimises the sum of
  squared orthogonal errors
\item
  We bend the principal component (for example) towards the data to
  achieve a better fit to the data
\item
  How far and how flexibly we can bend the curve towards the data is
  determined from the data to minimise a penalized criterion during
  fitting
\end{itemize}

\end{frame}

\begin{frame}{Principal Curves --- Comparison of estimation techniques}

\begin{center}\includegraphics[width=0.55\linewidth]{02-stratigraphic-data_files/figure-beamer/comparison-figure-3-1} \end{center}

\end{frame}

\begin{frame}{Principal Curves --- Fitting algorithm}

\begin{itemize}
\tightlist
\item
  Start with any smooth curve --- the first or second PCA or CA axis
\item
  Begin the \alert{Projection Step}

  \begin{itemize}
  \tightlist
  \item
    All objects are projected onto a point on the smooth curve that they
    are closest too
  \item
    The distances of the points along the curve that each object
    projects onto are determined
  \end{itemize}
\item
  Begin the \alert{Local Averaging Step}

  \begin{itemize}
  \tightlist
  \item
    Bend the current smooth curve towards the data so that the sum of
    squared orthogonal distances is reduced
  \item
    Taking each species (variable) in turn as the response, fit a
    smoother to predict the response using distance along the current
    curve as the predictor variable
  \item
    Repeat for all species (variables) and collect the fitted values of
    the individual smoothers into a matrix that described the new
    location of the curve in high dimensions
  \end{itemize}
\item
  If the new curve is sufficiently similar to the current curve, declare
  convergence
\item
  If algorithm has not converged, iterate the projection and local
  averaging steps until convergence
\end{itemize}

\end{frame}

\begin{frame}{Principal Curves --- Fitting algorithm}

\begin{center}\includegraphics[width=0.55\linewidth]{02-stratigraphic-data_files/figure-beamer/iterations-figure-1} \end{center}

\end{frame}

\begin{frame}{Principal Curves --- How Smooth?}

\begin{itemize}
\tightlist
\item
  The smoother fitted to produce the principal curve is a plug-in
  element of the algorithm
\item
  Can use any smoother; here used cubic regression splines
\item
  Important to not over fit the data by allowing too-complex a curve
\item
  Several options

  \begin{itemize}
  \tightlist
  \item
    Fit PCs with a large span (few df), note error, then reduce span
    (increase df), note error, etc. Use screeplot to determine optimal
    span
  \item
    Fit smoothers to each species using starting curve, allowing (G)CV
    to choose optimal smoothness for each species. Fit the PC using the
    median of the smoothness values over all species
  \item
    Allow the optimal degree of smoothness to be determined for each
    species individually during each local averaging step
  \end{itemize}
\item
  Advantage of latter is that those species that vary along the curve
  more strongly can use more degrees of freedom than those species that
  only vary lineally
\end{itemize}

\end{frame}

\begin{frame}[fragile]{Principal Curves --- Abernethy Forest}

\begin{Shaded}
\begin{Highlighting}[]
\NormalTok{>}\StringTok{ }\NormalTok{aber.pc <-}\StringTok{ }\KeywordTok{prcurve}\NormalTok{(abernethy2, }\DataTypeTok{trace =} \OtherTok{FALSE}\NormalTok{, }\DataTypeTok{vary =} \OtherTok{TRUE}\NormalTok{, }\DataTypeTok{penalty =} \FloatTok{1.4}\NormalTok{)}
\NormalTok{>}\StringTok{ }\NormalTok{aber.pc}
\end{Highlighting}
\end{Shaded}

\begin{verbatim}

    Principal Curve Fitting

Call: prcurve(X = abernethy2, vary = TRUE, trace = FALSE, penalty
= 1.4)

Algorithm converged after 6 iterations

           SumSq Proportion
Total     103234      1.000
Explained  98864      0.958
Residual    4370      0.042

Fitted curve uses 218.3391 degrees of freedom.
\end{verbatim}

\begin{Shaded}
\begin{Highlighting}[]
\NormalTok{>}\StringTok{ }\KeywordTok{varExpl}\NormalTok{(aber.pc)}
\end{Highlighting}
\end{Shaded}

\begin{verbatim}
      PrC 
0.9576693 
\end{verbatim}

\end{frame}

\begin{frame}{Principal Curves --- Abernethy Forest}

Visualise the fitted curve in PCA space

\begin{center}\includegraphics[width=0.5\linewidth]{02-stratigraphic-data_files/figure-beamer/plot-abernethy-prcurve-1} \end{center}

\end{frame}

\begin{frame}{Principal Curves --- Comparison with PCA and CA}

\begin{itemize}
\tightlist
\item
  The PC describes the long, sequential gradient in vegetation in a
  single variable
\item
  The PC explains 96\% of the variance in the absolute pollen data
\item
  PCA axis 1 explains 47\% and CA axis 1 31\% of the variance in the
  data
\item
  We need at least 2 PCA axes to fully capture the single gradient
  (80.2\%)
\item
  Distance along the curve between adjacent time points is a measure of
  compositional change
\item
  Can be expressed as a rate of compositional change --- highlights the
  periods of rapid compositional change in the Abernethy sequence
\end{itemize}

\end{frame}

\section{Rate of change analysis}\label{rate-of-change-analysis}

\begin{frame}{Rate of change analysis}

\begin{itemize}
\tightlist
\item
  Stratigraphic sequences record changes over time
\item
  How quickly do these changes take place?
\item
  Rate of change analysis aims to answer this
\item
  Two general approaches:

  \begin{itemize}
  \tightlist
  \item
    change in ordination units
  \item
    change measured in dissimilarity
  \end{itemize}
\item
  Could also use

  \begin{itemize}
  \tightlist
  \item
    derivatives of splines from principal curve
  \item
    derivatives of GAM(s) fitted to variables of interest
  \end{itemize}
\end{itemize}

\end{frame}

\begin{frame}{Rate of change analysis}

\begin{itemize}
\tightlist
\item
  Jacobsen \& Grimm (1988) method involves

  \begin{itemize}
  \tightlist
  \item
    smooth the data
  \item
    interpolate to constant time intervals
  \item
    ordinate smoothed, interpolate data (e.g.\textasciitilde{}using DCA)
  \item
    calculate change in ordination/axis score units as measure of RoC
  \end{itemize}
\item
  Dissimilarity-based approach can be performed two ways

  \begin{itemize}
  \tightlist
  \item
    Smooth the data \& interpolate, \emph{then} compute dissimilarity
    between interpolated levels
  \item
    Compute dissimilarity between adjacent samples directly, then
    standardise dissimilarity by time interval between samples.
  \end{itemize}
\end{itemize}

\end{frame}

\begin{frame}{Rate of change analyis}

TODO - needs an example

\end{frame}

\section{Chronological clustering
(zonation)}\label{chronological-clustering-zonation}

\begin{frame}{Chronological clustering}

\begin{itemize}
\tightlist
\item
  Chronological (or constrained) clustering commonly used to partition a
  sediment sequence into 2 or more zones
\item
  Useful for, \emph{inter alia}

  \begin{itemize}
  \tightlist
  \item
    delineating periods of similar species composition
  \item
    identifying discontinuities or periods of rapid change
  \item
    to facilitate description of a stratigraphic sequence
  \end{itemize}
\item
  As with standard cluster analysis, plethora of methods available

  \begin{itemize}
  \tightlist
  \item
    Optimal partitioning
  \item
    Binary (divisive) splitting
  \item
    Agglomerative partitioning
  \end{itemize}
\item
  Can be used with any dissimilarity (in theory), but common ones are

  \begin{itemize}
  \tightlist
  \item
    Cluster sums of squares (within-group Euclidean distance)
  \item
    Cluster-wise information statistic
  \end{itemize}
\end{itemize}

\end{frame}

\begin{frame}{Chronological clustering}

\begin{itemize}
\tightlist
\item
  Optimal partitioning

  \begin{itemize}
  \tightlist
  \item
    Identifies optimal locations for splits to form \(k\) zones
  \item
    Non-hierarchical, 3 zone solution \textbf{not} found by splitting
    one of the zones from the two zone solution
  \item
    Split placed to minimise within-cluster sum of squares or
    information content
  \end{itemize}
\item
  Binary (divisive) splitting

  \begin{itemize}
  \tightlist
  \item
    Similar to optimal method but \textit{is} hierarchical
  \item
    Split sequence into two zones, then split one of the 2 resulting
    zones, repeat
  \item
    Zone that is split is the one that would reduce within-group SS or
    IC the most
  \end{itemize}
\item
  Agglomerative partitioning

  \begin{itemize}
  \tightlist
  \item
    Start with all samples in separate zones and fuse the most similar
    adjacent samples
  \item
    Repeat, each time fusing most similar samples or zones
  \end{itemize}
\end{itemize}

\end{frame}

\begin{frame}{CONISS}

es the long, sequential gradient in vegetation in a single variable -
The PC explains 96\% of the variance in the absolute pollen data - PCA
axis 1 explains 47\% and CA axis 1 31\% of the variance in the data - We
need at least 2 PCA axes to fully capture the single gradient (80.2\%) -
Distance along the curve between adjacent time points is a measure of
compositional change - Can be expressed as a rate of compositional
change --- highlights the periods of rapid compositional change in the
Abernethy sequence

\end{frame}

\section{Rate of change analysis}\label{rate-of-change-analysis-3}

\begin{frame}{Rate of change analysis}

\begin{itemize}
\tightlist
\item
  Stratigraphic sequences record changes over time
\item
  How quickly do these changes take place?
\item
  Rate of change analysis aims to answer this
\item
  Two general approaches:

  \begin{itemize}
  \tightlist
  \item
    change in ordination units
  \item
    change measured in dissimilarity
  \end{itemize}
\item
  Could also use

  \begin{itemize}
  \tightlist
  \item
    derivatives of splines from principal curve
  \item
    derivatives of GAM(s) fitted to variables of interest
  \end{itemize}
\end{itemize}

\end{frame}

\begin{frame}{Rate of change analysis}

\begin{itemize}
\tightlist
\item
  Jacobsen \& Grimm (1988) method involves

  \begin{itemize}
  \tightlist
  \item
    smooth the data
  \item
    interpolate to constant time intervals
  \item
    ordinate smoothed, interpolate data (e.g.\textasciitilde{}using DCA)
  \item
    calculate change in ordination/axis score units as measure of RoC
  \end{itemize}
\item
  Dissimilarity-based approach can be performed two ways

  \begin{itemize}
  \tightlist
  \item
    Smooth the data \& interpolate, \emph{then} compute dissimilarity
    between interpolated levels
  \item
    Compute dissimilarity between adjacent samples directly, then
    standardise dissimilarity by time interval between samples.
  \end{itemize}
\end{itemize}

\end{frame}

\begin{frame}{Rate of change analyis}

TODO - needs an example

\end{frame}

\section{Chronological clustering
(zonation)}\label{chronological-clustering-zonation-1}

\begin{frame}{Chronological clustering}

\begin{itemize}
\tightlist
\item
  Chronological (or constrained) clustering commonly used to partition a
  sediment sequence into 2 or more zones
\item
  Useful for, \emph{inter alia}

  \begin{itemize}
  \tightlist
  \item
    delineating periods of similar species composition
  \item
    identifying discontinuities or periods of rapid change
  \item
    to facilitate description of a stratigraphic sequence
  \end{itemize}
\item
  As with standard cluster analysis, plethora of methods available

  \begin{itemize}
  \tightlist
  \item
    Optimal partitioning
  \item
    Binary (divisive) splitting
  \item
    Agglomerative partitioning
  \end{itemize}
\item
  Can be used with any dissimilarity (in theory), but common ones are

  \begin{itemize}
  \tightlist
  \item
    Cluster sums of squares (within-group Euclidean distance)
  \item
    Cluster-wise information statistic
  \end{itemize}
\end{itemize}

\end{frame}

\begin{frame}{Chronological clustering}

\begin{itemize}
\tightlist
\item
  Optimal partitioning

  \begin{itemize}
  \tightlist
  \item
    Identifies optimal locations for splits to form \(k\) zones
  \item
    Non-hierarchical, 3 zone solution \textbf{not} found by splitting
    one of the zones from the two zone solution
  \item
    Split placed to minimise within-cluster sum of squares or
    information conten
  \end{itemize}
\item
  Binary (divisive) splitting

  \begin{itemize}
  \tightlist
  \item
    Similar to optimal method but \textit{is} hierarchical
  \item
    Split sequence into two zones, then split one of the 2 resulting
    zones, repeat
  \item
    Zone that is split is the one that would reduce within-group SS or
    IC the most
  \end{itemize}
\item
  Agglomerative partitioning

  \begin{itemize}
  \tightlist
  \item
    Start with all samples in separate zones and fuse the most similar
    adjacent samples
  \item
    Repeat, each time fusing most similar samples or zones
  \end{itemize}
\end{itemize}

\end{frame}

\begin{frame}{Chronological clustering}

TODO insert HJBB's figure 1 from chpt 11 DPER5

\end{frame}

\begin{frame}{CONISS}

\begin{center}\includegraphics[width=0.7\linewidth]{02-stratigraphic-data_files/figure-beamer/coniss-1-1} \end{center}

\end{frame}

\begin{frame}{Binary splitting via MRT}

\begin{center}\includegraphics[width=0.9\linewidth]{02-stratigraphic-data_files/figure-beamer/mrt-partykit-abernethy-1} \end{center}

\end{frame}

\end{document}
